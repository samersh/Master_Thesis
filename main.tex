\documentclass [12pt, oneside] {book}

\usepackage{amssymb}
\renewcommand{\baselinestretch}{1.5}
\usepackage[utf8]{inputenc}
\usepackage{geometry}
\geometry{a4paper, portrait, margin=2.5cm}
\usepackage[round]{natbib}
\usepackage{graphicx} %package to manage images
\usepackage{dirtytalk}
\graphicspath{ {./images/} }
%\usepackage[rightcaption]{sidecap}
\usepackage[font=scriptsize, labelfont=bf]{caption}
\usepackage{wrapfig}
\usepackage{hyperref}
\usepackage[acronym, toc]{glossaries}
\usepackage{acronym} 
\usepackage[nottoc]{tocbibind}
\makeglossaries


\begin{document}
\begin{titlepage}
    \begin{center}
        \vspace*{1cm}
        \Huge
        \textbf{Action Research in Virtual Reality: Glimpses from Palestine Application (Virtual Time-Machine Tour)}
        \vfill
        \LARGE
        \textbf{Master Thesis}
        
        \vspace{0.5cm}
        submitted by:
        \vspace{1.5cm}
        \textbf{Samer Shawar}\\
        Matriculation Number: 1380673
        \vfill
        \textit {First Examiner: Prof. Dr. Volker Wulf}
        
        
        \textit {Second Examiner: Dr. Markus Rohde}
        \vfill
        %\vspace{0.8cm}
        \Large
        Human Computer Interaction\\
        University of Siegen\\
        December 2019\\
 \end{center}
\end{titlepage}
\pagenumbering{Roman}
\addcontentsline{toc}{section}{Acknowledgement}
\section*{Acknowledgement}
bla bla bla bla
\newpage
\section*{Abstract}
Since 1948, the Palestinian refugees are not allowed to go back to Palestine. The existence of the Israeli state prevented the Palestinian refugees to get back to their homes. Their houses and villages were depopulated and demolished. Traveling to Palestine could be intimidating for some people due to the perception that the western media have shown about the country. Palestinian refugees need to have access to see Palestine. Anybody who wants to visit the country, but has political situation concerns, must experience or see what to expect of a visit to Palestine. Action research was the concept of developing Glimpses from Palestine \acrlong{vr} application. The application development was based on the insights that were collected from interviews. The rebuilding of one of the demolished villages in \acrlong{vr} has given the application a historical and political aspect. The results show a strong connection between the second and third generations of Palestinians and Palestine. The transition of stories from grandparents to grandchildren in the diaspora and their effect on maintaining a strong relationship with the country. Also, the results showed the effect that Glimpses from Palestine has on people to preserve the value of their country and villages. This research explores the \acrlong{vr} as a new approach for documenting and displaying the historical events and facts of a country as Palestine, that the history and the demography of the country were changed due to political conflict. 
\addcontentsline{toc}{section}{Abstract}
\tableofcontents
\listoffigures

\newacronym{vr}{VR}{Virtual Reality}
\newacronym{ui}{UI}{User Interface}
\newacronym{yallah!}{YALLAH!}{You all are hackers!}
\newacronym{idf}{IDF}{Israel Defence Forces}
\newacronym{hmd}{HMD}{Head-Mounted Display}
\newacronym{pd}{PD}{Participatory Design}
\newacronym{pa}{PA}{Palestinian Authority}
\newacronym{unrwa}{UNRWA}{United Nations Relief and Works Agency for Palestine Refugees}
\newacronym{plo}{PLO}{Palestine Liberation Organization}
\newacronym{un}{UN}{United Nations}
\newacronym{jnf}{JNF}{Jewish National Fund}
\newacronym{ala}{ALA}{Arab Liberation Army}
\newacronym{fov}{FOV}{Field of View}
\newacronym{for}{FOR}{Field of Regard}
\newacronym{cave}{CAVE}{Cave Automatic Virtual Environment}
\newacronym{cad}{CAD}{Computer-Aided Design}
\newacronym{dof}{DOF}{Degrees of Freedom}
\newacronym{nfc}{NFC}{Near Field Communication}
\newacronym{cam}{CAM}{Computer-Aided Manufacturing}
\newacronym{cae}{CAE}{Computer-Aided Engineering}
\newacronym{sdk}{SDK}{Software Development Kit}
\newacronym{svg}{SVG}{Scalable Vector Graphics}
\newacronym{png}{PNG}{Portable Network Graphics}

\printglossary[type=\acronymtype, title=Abbreviations, nonumberlist]


\chapter{Introduction}
\pagenumbering{arabic}
During the last decade the word ‘virtual’ became one of the most exposed words in the
English language, today we have virtual universities, virtual offices, virtual pets, virtual
studios, etc., and all of this because of virtual reality \acrshort{vr} \citep{Vince2011}. The \acrshort{vr} hit the
headline in the mid-1980’s and spawned a series of conferences, exhibitions, television
programs, and philosophical debates about the meaning of reality (Vince, 2011). “Virtual
reality is using a computer to create images of 3D scenes where the person can navigate and
interact” (Vince, 2011). Pinotti (2017) explained that immersivity and interactivity in virtual
environments are able to elicit in the user an intense feeling of “being there”, namely of being
embodied in an independent and self-referential world (Pinotti, 2017).

This master thesis will examine a virtual reality smart-phone application if it can let people overcome borders and if virtual reality can strengthen resilience for people by seeing their rebuilt villages in the virtual world. 


Virtual reality will need the development of the environment, according to the previous study at the Hackathon exchange program the users showed the interest of people for having a smart-phone application that allows them to experience the virtual reality at Palestine. For developing the application framework, a development environment like Unity Gaming engine will be used for improving the application due to the easy implementation of the Virtual Reality environment at Unity and the ability to distribute it over a variety of devices. Unity provides the ability to develop a virtual reality environment and Implement 3D models inside the real 360$^{\circ}$ videos.


Respectively for the maps and the data collected about the structure of the villages, the houses will be 3D modeled to rebuild a village or a part of it in the Virtual world. Blender will be used to create the 3D models of the houses and texture the buildings in the villages. Blender is an open-source 3D animation software, and it is structured for modeling and animation. The 3D modeled village will be integrated within the 360$^{\circ}$ video of the same village area. Points of interest will be distributed among the map of the village and the 360$^{\circ}$ video will be filmed from the same points of interest. The Villages actual locations will be filmed by a GoPro Fusion a 360$^{\circ}$ 5K camera. Through the prototype, the user will be able to see the actual location of the village nowadays, and will also have the option to see how was it pre-1948. 

The sound of the surrounding environment will immerse the user into the scenes, for interactivity between the user and the virtual environment, the user will have a gaze feature that allows commuting between the points of interest inside the village for instance. The gaze feature will be activated only when the user looks at the point of interest for a few seconds, a time-line will start moving to inform and prepare the user to be transferred to another scene. The user needs a virtual reality glasses to be able to get immersed in the application and see the whole environment. The virtual reality glasses are expensive and they need a very good computer to be able to operate them. Google created a VR lens out of cardboard, it allows the users to use their smart-phones as a display for the cardboard and experience virtual reality. The cardboard is easy to move with it and pass different checkpoints, airports, borders because it is only cardboard and it is also very cheap. That will make it easy for us to move around and show the application to different people and evaluate the application.

Participatory action research is best represented as a self-reflective
spiral of cycles of planning, acting and observing, reflecting and then re-planning in successive cycles of improvement \citep{KemmisSdanMcTaggart1988}. Participatory action research will allow us to evaluate the application with the participants and understand their concerns about their usability and interaction with the design. we will give the participants the prototype and let them test it, while we are observing how they interact and use the prototype, also we will let them reflect on the design and the product and take it for improvement in the design. Participatory action research will allow us to have a better evaluation from users to form a better action and interaction where the users had already conducted practice on the product. The evaluation will be taken by different participants from Palestinian refugees in Palestine, Jordan, and Lebanon, but also non-Palestinians will evaluate the application since the application is addressed for all people and see if Virtual reality can let them overcome borders.





The idea is to develop a new application to show Palestine with 360$^{\circ}$ videos, but it will include a part of one of the demolished villages before 1948. To rebuild one of those villages we need to know the structure of the village and how it looked like. The data about the structure of the village will be collected through personal stories, old pictures, British mandate aerial photos, and some maps made by a few people who lived in those villages. 
  
  
In the master project, a pre-study of the project will be conducted with Palestinians who live in the diaspora through interviews and workshops.  The Interview is a highly used method of collecting data in qualitative social research methods” \citep{Anyan2013}.  Kvale(1983)  described the purpose of as a method collection in social research as “... to gather descriptions of the life-world of the interviewee with respect to interpretation of the meaning of the described phenomena”  \citep{Kvale1983}. Semi-structured interviews will be used for collecting stories and exploring personal experiences about the Palestinians displacement and their relation with Palestine after living abroad. A questions guideline will be prepared for the interviews with  open-ended questions allowing a discussion with the interviewee rather than  formulated questions and answers.

After the interview some insights would be collected from the interviewee through a participatory design workshop. Participatory design (PD) is a set of theories, practices, and studies related to end-users as full participants in activities leading to software and hardware computer products and computer-based activities  \citep{Muller2003}. The field is extraordinarily diverse,  drawing on fields such as user-centered design, graphic design, software engineering, architecture, public policy, psychology, anthropology, sociology, labor studies, communication studies, and political science, and from localized experiences in diverse national and cultural contexts \citep{Gregory2003}.  Personal stories might give the ability for some refugees to draw the houses of their villages how they remember it. Applying a workshop for some of the Palestinian refugees to re-draw their houses would demonstrate a better image for the construction of the village, and that will give them the ability to tell us what they would like to see. houses, public places, walking paths between the houses, farms...etc. 
Information and communications technologies (ICT) artifacts should react to changing conditions of a social system \citep{Wulf2011}. Design case studies give us the ability to understand the relationship between social practice and the design of ICT artifacts that are built to support those practices, It is not so much about high tech, but about high-value \citep{Volker2013}. Design Case Studies ideally consist of three phases: (1) Analyze and particularly describe a social practice on a micro-level, as well describing the tools, media and their usage. (2) Describe the ICT artifacts from a product and from a process perspective, like the design concepts, and the applied design methods. (3) Document the introduction, appropriation and the potential re-design of the ICT artifact, documentation allows analyzing the trans-formative impact of certain functions and design options \citep{Volker2013}. Design case studies will allow us to check with the Palestinian refugees the tools or the practice that they do to see or to feel the connection with Palestine as a country and with their villages. After understanding their tools or practice, we can build a design depending on a specific design concept and methods. By letting them trying the prototype we will have general feedback about the interaction of the user with the App, and what would need to maintain and what kind of things needs to be re-designed. Design case studies will give us a general overview of the design concept and full fill us with the answers for the "how" and "why" about using the design methods.   
The Application will start by showing an overview of the whole map of historical Palestine. Points of interest will hover over the cities, the user will select a point of interest through the gaze feature. After selection, the map will be zoomed to the city location and will show more specified places within the city and the villages surrounding the city. The user will be transferred to a 360$^{\circ}$ video of the desired place, and inside the 360$^{\circ}$ video the user can commute inside the village for instance through the points of interest. sound of the surrounding environment will be added to the videos, as well as narrating the history and personal stories about the place. Within the demolished village, the user will have the ability to see how is the village look like now, and how it used to look like before 1948 by selecting an interactive button to travel in time. The user will have the ability to move back to the main map to choose to see different locations. All the methods that are used will help us to develop and collect the information and the data that is needed for the application and to be presented to the user.

\chapter{State of the Art}
We will define \acrfull{vr} and show the main characteristics of it through this chapter. Followed by listing a number of related projects that has been made in different places to immerse and show the user locations in different times. The last section will represent example projects that has been done in \acrshort{vr} in different places in the world focused immigration, refugees and struggles.   

\section{Virtual Reality}
Ivan Sutherland created a Head-Mounted Display in 1968 which displayed left and right views of a computer-generated 3D scene to the user, so that the digital scene remained static when the user's head was shifted. The images, as they were simple line drawings, were far from life. But as they were stereoscopic, the user had the impression of looking at a solid 3D object, which was the birth of Virtual Reality. In 1965 he published a paper entitled "the ultimate display" describing how computers would one day open a window to the virtual world for the user. \citep{Vince2011}.  

Virtual reality is a technology that is often regarded as a natural extension of 3D computer graphics with specialized input and output tools \citep{Jayaram1997}. Ryan (2001) defined it as an “interactive, immersive experience generated by a computer” \citep{Ryan2001}. And according to G.C. Burdea and Coiffet (2017) “it is a generated computer graphics used to create a realistic-looking world that responds to the user’s input (gestures, verbal commands, etc.)” \cite[p.20]{burdea2017virtual}. Instead of seeing a screen in front of them, the users will be engulfed in a 3D environment by virtual reality.“The scientific community has been working in the field of virtual reality (VR) for decades, having recognized it as a very powerful human-computer interface”\cite[p.19]{burdea2017virtual}. 

As shown in Figure \ref{fig:3I} In order to build a situation of virtual reality, three elements are needed:  immersion, interaction, and imagination. They are called the “3I’s” of virtual reality \citep{Hu2016,burdea2017virtual}.

\begin{wrapfigure}{r}{0.30\textwidth} %this figure will be at the right
    \centering
    \includegraphics[width=0.28\textwidth]{3I}
    \caption{The 3I's of Virtual Reality - © 2003 by John Wiley \& Sons Inc. All rights
reserved}
    \label{fig:3I}
\end{wrapfigure}



1. \textbf{Immersion:} it is the virtual reality situation where the user feels personally inside the
scene and immerse inside the simulated virtual world.



2. \textbf{Interaction:} The interactive feedback between the user
and the virtual environment. Since it is a man-machine
interface, the system should promptly respond to the
user’s actions.




3. \textbf{Imagination:} The scene design and the construction of
the environment formulated with imagination for a
better user simulated experience.

\section{Types of Virtual reality systems}
We categorized the VR technologies into three main categories, Non-immersive system, semi-immersive system, and Immersive system, according to \cite{Bamodu2013VirtualComponents}. This is based on the system's level of immersion, interface, and components used.  

\textbf{Non-Immersive VR system}
It is also called the World system's Desktop VR system, Fishtank or Window is the least immersive and least expensive of the VR systems because it needs the least complex components. This uses traditional graphics workstations with a screen, a keyboard, and a mouse, with modeling and CAD systems in its software areas.  

\textbf{Semi-Immersive VR system}
Use a relatively high-performance graphics computing system coupled with a large surface to display the visual scene, provide high immersion levels while maintaining desktop VR simplicity or using some physical model. One such program is the CAVE (Cave Automatic Virtual Environment) and the driving simulator is an application. 

\textbf{Immersive VR system}
 The immersive VR system is the most expensive and provides the highest level of immersion, its components include HMD, tracking devices, data gloves, and others that include the user with computer-generated 3D animation that gives the user a sense of being part of the virtual environment. One of its features is digital house walk-through \citep{Bamodu2013VirtualComponents, Baus2014MovingReview}.

\begin{figure}[ht]
    \centering
    \includegraphics[width=0.98\textwidth]{images/vrsystem.png}
    \caption{VR Systems A: non-immersive system, B: semi-immersive system, C: Immersive system \citep{Baus2014MovingReview}}
    \label{fig:vrsys}
\end{figure}




\section{Presence in virtual reality} 
Due to a strong mediated presence, emotional and physical reaction would be caused to the user as if the virtual world existed physically, the impression of being in a real place, as well as the illusion of plausibility, having the feeling that the scenario is actually taking place. Despite the fact that the user is aware that the virtual world is just a simulation, these illusions happen. \acrshort{vr} is one of the major trends in presence evolutione \citep{Waterworth2014, Steinicke2016}.
The Human-computer interaction is extended from purely visual interaction to diversified interaction that the user could interact with objects in virtual reality with the perceived experiences and cognitive processing abilities as in the real world, and the feeling is close to the changes in the natural world \citep{Hu2016}.
\acrshort{vr} gives the user the ability of controlling, moving and looking in the virtual world, that gives a significant effect on the level of presence, therefore the more controlling the user would have the greater the feeling of immersion\citep{William}. \say{When we feel strong mediated presence, we react emotionally and
bodily (at least to some extent) as if the virtual world existed physically}\cite[p.4]{Waterworth2014}.


To achieve a significant immersion level for the user, we need to understand the \acrfull{fov} and the \acrfull{for} for a human. The \acrshort{fov} for a human is approximately 200 degrees with a 120 degrees of binocular overlap. The display's field of view is a measure of the angular width of a user's vision that is covered by the display at any given time as shown in Figure \ref{fig:field}. In a three-sided \acrshort{vr} projection the field of view is reduced due to the edge of the screen but it is in 100\% if the user is looking forward. However, the stereo overlap is in a larger scale at the three-sided stationary projection, some \acrlong{hmd}'s provides a 100-120 degrees of \acrshort{fov} that is a reasonable portion of the human visual rang. Although the stereo overlap is very important and that can vary in \acrshort{hmd}'s where the two displays are more of less widely separated, if the overlapping  
\acrshort{fov} for both eyes is as small as 30 degrees, it will be difficult to perceive stereopsis \citep{William}.
If we move towards an object, the ciliary muscles adjust the shape of the lens to accommodate
the incoming light waves to maintain an in-focus image. Also, the eyes automatically
converge to ensure that the refracted images fall upon similar areas of the
two retinas. This process of mapping an image into corresponding positions upon the
two retinas is the basis of stereoscopic vision. The difference between the retinal images
is called binocular disparity and is used to estimate depth, and ultimately gives rise to
the sense of three dimensional\citep{Vince2011}.

\begin{figure}[ht]
    \centering
    \includegraphics[width=0.90\textwidth]{Fod.png}
    \caption{Field of View - \citep{William}}
    \label{fig:field}
\end{figure}


\begin{figure}[ht]
    \centering
    \includegraphics[width=0.90\textwidth]{fov.png}
    \caption{Field of Range - \citep{William}}
    \label{fig:fod}
\end{figure}

The \acrfull{for} is the amount of space that is surrounding the user inside the virtual world. As displayed in Figure \ref{fig:fod}, the \acrshort{for} in the \acrshort{hmd} is 100\% due to the screens that always face the user's eyes, regardless for the direction that the user will look at, the virtual world will always be in front of his eyes. The \acrshort{for} is usually less than 100\% in the three-sided \acrshort{vr} projection because the virtual world will always need screens and it not dynamic in movement as the \acrlong{hmd} \citep{William}.

A \acrshort{hmd} is nothing more than a stereoscope, it doesn't use only photographs but also animated
video images. What is essential though, is that the views seen by the two eyes
must overlap and it is in this area of overlap that stereoscopic vision is perceived \citep{Vince2011}.
Figure \ref{fig:field} presents this stereo-overlap area. As the two images are taken from two different positions in space, the brain uses these
differences to create a single image that contains depth information \citep{Vince2011}. Therefore, the use of \acrshort{hmd}'s would have its consequences, for instance a low-latency tracking "not more than 2ms (millisecond)" is desired to adapt the display imagery to the physical movement of the viewer, also to avoid motion sickness since large latency's have serious negative effects on the simulation, for example, headaches, nausea, or vertigo \citep{burdea2017virtual, Vince2011, Steinicke2016}.

A technological revolution was made by smartphones where they combined between communication and computation devices, where they are small by size and can fit in the user pocket. the high computation processing in the smartphones allowed is enough to process \acrshort{vr} environment, therefore it enabled a light-weighted and practical \acrshort{vr} devices and the resurgence of the internet in \acrshort{vr}. Today’s principal main components 
of smartphones such as high-density display panels, gyroscopes, or accelerometers 
are built in most devices and therefore they cost only a fraction of the price of 
Virtuality machines in the early 1990s \citep{Steinicke2016}. 



  
\section{Related Work}
The SCULPTEUR project – Semantic and content-based multimedia exploitation for European benefit – is developing a solution for museums to create and manipulate digital representations of museum objects. SCULPTEUR utilizes multi- stereo and silhouette techniques to create 3D museum object reconstructions stored in a database together with other multimedia data. The project is also aimed at developing a semantic layer providing a variety of search and content analysis opportunities. Moreover, the system gives users seamless access to the distributed databases of cultural data by offering a set of tools encompassing educational software products [SCULPTEUR 2003].
The 3D Murale project – 3D Measurement and Virtual Reconstruction of Ancient Lost Worlds of Europe – is aimed at developing a system capable of recording archaeology excavation phases using Virtual Reality techniques. In addition to the artifacts also stratigraphical layers can be modeled. This requires utilizing diverse 3D capture techniques. Furthermore, the project offers the reconstruction of excavated remains of pottery, sculptures and buildings as well as their visualization in a way as they possibly looked like throughout ages [3D Murale 2003].
In recent research on exploitation of modern technologies in cultural heritage, an increasing role of Augmented Reality can be observed. A notable example is the ARCHEOGUIDE project intended to develop a wearable AR tour guide at cultural heritage sites [ARCHEOGUIDE 2003]. ARCHEOGUIDE allows visitors to see virtual reconstructions of ancient buildings. A user is equipped with a see-through Head-Mounted Display (HMD) and wearable computing equipment. The equipment is responsible for visualization of appropriate information on the HMD depending on the visitor’s position and orientation in the site. Similar to ARCHEOGUIDE in terms of applied technology is the LIFEPLUS project. A fundamental difference is the presented contents that in case of LIFEPLUS additionally encompass real- time 3D simulations of ancient fauna and flora [LIFEPLUS 2003].
Another example of an outdoor solution is the Ename 974 project aimed at virtual reconstruction of an extensive archaeological site located at Ename, Belgium [Ename 974 2003]. Visitors are offered with virtual reconstruction of early-medieval buildings in stationary AR kiosks. Users can see visualizations of virtual models overlaid on video data captured by a camera. Using touch screen displays they can control the camera and the displayed data.
Another group of systems exploiting AR techniques to visualize cultural heritage encompasses indoor applications. An example of such a system is the Virtual Dig Experience installed in the Seattle Art Museum [Virtual Dig 2003]. Using VR and AR techniques visitors, particularly children, can discover artifacts for themselves. Users are presented not only with the artifacts but also with their archaeological context. Real objects such as brushes and small shovels are used for user interaction. The Virtual Dig has been developed based on the HI-SPACE [HI- SPACE 2003] and ARToolKit [ARToolKit 2003] packages.
The Virtual Showcase project [Bimber et al. 2003; Virtual Showcases 2003] is intended to eliminate the necessity of using unusual methods of presenting cultural objects, which is one of the weaknesses of most VR and AR systems. The project aims at integrating the AR technology into a traditional museum showcase format. Virtual images can be projected on the sides of a specially designed showcase allowing users to explore both the real objects and their virtual representations.
\citep{Rahaman2011InterpretingPerspective}

\section{Example projects}
\textbf{Auschwitz Virtual Tour}\footnote{Auschwitz VR Tour-\url{https://youtu.be/EOM_CxAKB_Y}}: The German broadcasting institution WDR made a 360$^{\circ}$
documentary in Auschwitz concentration camp. Within the documentary, some Holocaust
survivors tell their stories. While listening to the stories and seeing the different locations in
the camp the user could feel the fear and horror that people suffered from. The video
immerses the user through the sound of the surrounding environment in the camp, you can
hear the wind and the sound gives a slight feeling of the cold weather over there. The
experience is immersive, but there is no interactivity with the user.


\textbf{Clouds over Sidra}\footnote{The Za’atari camp VR Tour-\url{https://youtu.be/mUosdCQsMkM}}: A 12 years old girls’ daily life story at the Za’atari refugee camp in Jordan
showed in virtual reality. The camp is a home for 80,000 Syrian refugees, half of them are
children(“Syrian Refugee Crisis – UN Virtual Reality,” 2015). The documentary was made by
the United Nation to raise awareness about the Syrian crisis. The video contained a number
of short videos from different parts of the camp, and it’s being synchronized while Sidra
narrates her story. It is more like watching a video with empathy than being immersed, but
the video presents the real life of the camp. Although the difference that while watching and
listening to the story, the user observes the people how they actually survive and live in the
camp. Therefore, the user does not have to imagine how is life over there.


\textbf{Carne y arena}\footnote{Carne y Arena Trailer-\url{https://youtu.be/zF-focK30WE}} : A highly professional
Virtual reality project that puts viewers
into the harsh life of an immigrant. The
user is placed among a group of
Mexican immigrants passing the
borders into the U.S. It was written and
directed by Alejandro G. Iñárritu. It Is a
full virtual reality experience; the user
needs to reserve an individual session
on the website. According to Pinotti (2017) you go in a dark room; your feet are on the sand (coarse grain, rough feeling) then, two assistants welcome you and provide you with the
necessary devices: an Oculus Rift headset, a backpack connected via cables to a powerful
computer and you are ready to be caught up in a nightmare \citep{Pinotti2017}. The project is
subtitled by ‘Virtually present, Physically invisible’. Pinotti (2017) defined, Virtually present:
“you are transported in the middle of the desert, among men, women, and children who try
their voyage of hope”. Physically invisible: “you are present, but nobody sees you and after a
while, you start to feel the need to be noticed and seek acknowledgment of social
recognition” \citep{Pinotti2017}. The project was developed with high technology, like 3D
modeling, visual effects, sound. The interaction of the user and the feeling of “being there”
by placing the user in a special environment, leads to a unique experience of immersion.



\chapter{Background}
A chronology of key events in the history of Palestine and the development of the Israeli/Palestinian conflict since the 1880s. The political history is detailed to present the reasons for demolishing a large number of Palestinian villages. Focused details about Al-Ghabisiyya village as a case study. Then the chapter will address the cultural identity of the Palestinian refugees from the perspective of a second and third generation. The last section will explain the first steps of the application development from the \acrshort{yallah!} hackathon exchange program.  
\section{Political History}

Palestine, a nation known for decades to have existed in political conflict. In the 1880s, Zionist immigration to Palestine began with the goal of establishing a national home for the Jewish people on their ancient land, Israel, in Zionist words \citep{Morris2004, Pappe2006, Khalidi2015}.

After Basel's first Zionist Congress in 1897, at which the idea of creating a Jewish state in Palestine was first presented, Vienna's rabbis sent two delegates to examine the country's suitability for such an undertaking. 
In this document, the respondents reported to Vienna the results of their explorations:





\centerline{\textit{\say{The bride is beautiful, but she is married to another man.}}}


This document summarized the issue the Zionist movement had to confront from the start, an Arab nation actually living on the land the Jews had set their hopes upon. With the exception of some minority parties, the Zionist movement tended to ignore the existence of Palestinians over the land \citep{Shlaim2014, Karmi2007}.

\say{Zionism secularized and nationalized Judaism. To bring their project to
fruition, the Zionist thinkers claimed the biblical territory and recreated,
indeed reinvented, it as the cradle of their new nationalist movement. As they
saw it, Palestine was occupied by \say{strangers} and had to be repossessed.
\say{Strangers} here meant everyone not Jewish who had been living in Palestine
since the Roman period. For many Zionists Palestine was not even an 
\say{occupied} land when they first arrived there in 1882, but rather an \say{empty} 
one, the native Palestinians who lived there were largely invisible to them or, 
if not, were part of nature's hardship and as such were to be conquered and 
removed. Nothing, neither rocks nor Palestinians, was to stand in the way of 
the national \say{redemption} of the land the Zionist movement coveted} \cite[p.11]{Pappe2006}.


At the very same time, throughout the last decades of the 19$^{th}$ century, the Arab Revolution was emerging to achieve \say{independence} from the Ottoman Empire. Nevertheless, the possibility for a possible Jewish conquest of the state and the displacement of the native Palestinian people was clear to some Palestinian leaders even before the First World War, it was acknowledged in the publications of the founders of Zionism. Historical evidence indicates that between 1905 and 1910, many Palestinian leaders addressed Zionism as a political movement to buy property, and influence in Palestine, although the disruptive aspect was not well understood at the time. \citep{Pappe2006}.



 The British government published a Balfour declaration in 1917 and during the First World War declaring the support for creating a \say{National Home for the Jewish people in Palestine}. The First World War devastated the Ottoman Empire that same year, and Britain captured Palestine \citep{Morris2004}.  
 
 
 
 
 
 The British Mandatory for Palestine was accepted in 1923 and from that year until 1948 Palestine was under the British mandate. The British Mandatory Government officials had permitted the Zionist movement to create its autonomous territory within Palestine with an infrastructure for a future state. By the late 1930s, the Zionist leaders were able to shape the theoretical concept of Jewish permanence into concrete plans \citep{Pappe2006}.  
 
Orde Charles Wingate, an officer of the British Army who made the Zionist leaders understand more thoroughly that the concept of Jewish statehood will have to be strongly associated with militarism and an army, Wingate also became a Zionist and began to train and instruct Jewish settlers techniques of fighting against the native population, in 1920 he founded and created \say{Hagana} \textit{(“Defence” in Hebrew)}, the Jewish community armed group in Palestine. During the Arab revolt against the British Mandate, he succeeded in merging Hagana troops with British forces. The Hagana troops received their first training in 1938 to capture a Palestinian village together with the British forces assaulted a village at the Lebanese border and retained it for hours \citep{Pappe2006}. \say{Moshe Dayan \textit{(an Israeli military leader and politician)}, who said: Wingate had taught us everything we know}\cite [p.112]{Fenby2018}. 

In the Second World War, Hagana has gained valuable combat experience when many of its participants served for the British war effort \citep{Pappe2006}. Irgun and Lehi \textit{(Stern Gang)} two paramilitary groups that came out of Hagana. Irgun split from Hagana in 1931 and Lehi divided from Irgun in 1940\citep{Shlaim2014}. Both groups were seen as terrorist organizations or entities which commits acts of terror against Palestinians and British authorities \citep{Bell1976}.   


The British government released a policy paper entitled \say{The White Paper} in May 1939, promising the inhabitants of Palestine statehood and autonomy within ten years, and restricting Jewish immigration to 75,000 for five years. The White Paper also dismissed the notion of dividing Palestine in order to stop the Arab rebellion that began in 1936 \citep{Morris2004, Fenby2018}.

That wasn't in the Jewish Agency's benefit. The Jewish Agency, therefore, started to consider military intervention as tension mounted to strike the British within the Haganah. The Haganah continued to remain with the British in cooperation. But in 1944, the Irgun and Lehi started a revolt toward British rule. In reaction to British restrictions on immigration, they targeted police and government targets.
Certainly, Lehi(Stern Gang) was determined to fulfill the Zionist dream, so Stern, before he realized that Hitler was exterminating the Jews in Europe, has offered Nazi Germany help and support to fight the British and push them out of Palestine. This initial relationship with Nazi Germany eventually cost Lehi and Stern himself a great deal of support \citep{Shlaim2014, Heller1995, Grob-Fitzgibbon2011}. 

Considering the reality of exterminating the Jews in Nazi Germany concentration camps, Irgun and Lehi intentionally avoided strategic targets to guarantee that the British war effort against their common enemy, Nazi Germany, was not hampered. After the 1945 defeat of Nazi Germany at World War II, the Haganah, Irgun, and Lehi decided to join as the Jewish Resistance Movement, they operated under a joint command formation consisting of members of all three groups and carefully planned their activities. The Haganah also assisted the Irgun command with 460 Palmach warriors (Haganah's elite armed forces) and provided it with funding. Whereas the Irgun and Lehi would continue to pursue a complete-scale rebellion towards the British, the Haganah envisaged a more targeted effort to force the British to adhere to Zionist demands, combining attacks primarily to immigration-related goals \citep{Bell1976, Shlaim2014}.

Notwithstanding the white paper orders, after the Holocaust Jewish illegal immigration to Palestine persisted and increased, the Zionist organizations in Europe set up a well-organized system to immigrate 350,000 Jewish immigrants to Palestine between November 1931 and December 1946.\citep{Grob-Fitzgibbon2011, Heller1995}.   

Lehi and Irgun expanded their terrorist attacks against the British authorities at various locations in Europe and Palestine, exploded the British Embassy in Rome, bombed the British Colonial Club in London, targeted the British 6$^{th}$ Airborne car park in Tel Aviv, destroyed the British police station in Haifa with a truck bomb and exploded the King David hotel \citep{Bell1976}. However the British reacted mildly, particularly when compared to the cruel treatment they conducted on Palestinian revolutionaries in the 1930s \citep{Pappe2006}.
 
 
 
 
\say{The decision was made by the British Cabinet to pull out of Mandatory
Palestine and leave it to the \acrshort{un} to solve the question of its future. The \acrshort{un}
took nine months to deliberate the issue, and then adopted the idea of partitioning
the country. This was accepted by the Zionist leadership who, after
all, championed partition but was rejected by the Arab world and the Palestinian leadership, who instead suggested keeping Palestine a unitary
state and who wanted to solve the situation through a much longer process
of negotiation}\cite[p.40]{Pappe2006}.
 
In the meantime, the Jewish paramilitary received a significant amount of weapons in a secret delivery in April 1948 \citep{Morris2008, Pappe2006}. 
The Haganna leaders formulated the strategy of ethnic cleansing at the beginning of March and it was named Plan D \citep{Shlaim2014}. The strategy targeted at removing every Arab or Palestinian aspect within the country \citep{Pappe2006, Shlaim2014}.
 
Plan D had multiple operations at various dates and different locations:
Nahshon in Jerusalem (2–3 April – 20 April),
Mishmar Ha‘emek (4–15 April),
Ramat Yohanan (12–16 April),
Arab Tiberias (16–18 April), 
Arab Haifa (21–22 April,)
Operation Yiftah, in eastern Galilee (15 April – 15 May), and
Operation Ben-Ami (parts I and II), in western Galilee\citep{Morris2004}.

The British departed on 15 May 1948 and the Jewish Agency proclaimed the creation in Palestine of a Jewish state officially recognized by the two superpowers of the day, the United States and the USSR \citep{Pappe2006}.
 
 According to \cite{Sanbar2007} “1948. That year, a country and its people disappeared from both maps and dictionaries.” \citep{Sanbar2007}. 



The war that led to the establishment of the Israeli state over Palestine in 1948 also contributed to the destruction of the Palestinian nation \citep{Sadi2007}. That day Israel was established over Palestine was a catastrophe for Palestinians \textit{(in Arabic, "Nakba")}. At least 80 percent of the Palestinians living in the territory of Palestine on which Israel was founded became refugees. “The lives of the Palestinians at the individual, community, and national level were dramatically and irreversibly changed” \citep{Sadi2007}.

A huge number of Palestinians were deported outside Palestine during the "Nakba". “Of the
1,400,000 Palestinians in the country prior to the Nakba, just 150,000 individuals were listed
as being present during the first census carried out by the new Israeli state” \citep{Sanbar2007}.



According to \cite{Falah1996}, Some of the terms of Israel's military code are Matate (Broom) and Bi'ur Chametz (Passover Cleanup) indicate that the war contained an element of ethnic cleansing \citep{Falah1996, Pappe2006}. During the 1948 war, Palestinian villages were destroyed and depopulated by Israeli forces, as shown in figure \ref{fig:map}, the red dots represent the demolished villages, and the yellow dots represent the depopulated villages. 
According to the \acrshort{unrwa}, more than 5 million Palestinian refugees were registered in January 2015, divided between, The West Bank, Gaza
Strip, Syria, Lebanon, and Jordan \citep{Khalidi2015, DajaniDaoudi2011}. 

The Palestinian refugees have not been allowed back to Palestine since 1948, and not even for a visit. Because of the idea of returning large numbers of Palestinians to their villages and cities, or even to any part of Palestine, it impacts on Israelis high-seated concerns about the validity and continuity of the whole Zionist system, and also the Arab-Jewish population equation in Palestine \citep{Khalidi2016}.

\begin{wrapfigure}{r}{0.20\textwidth} %this figure will be at the left
    \centering
    \includegraphics[width=0.20\textwidth]{d_villages}
    \caption{Demolished and depopulated villages - Palestine Open Maps, © 2018  Visualizing Palestine}
    \label{fig:map}
\end{wrapfigure} 

According to Pappé (2006), the Zionist organization and the Hebrew University mapped and made a detailed registry of all the Arab villages \citep{Pappe2006}.



The project of mapping the whole villages details was funded by \textit{\acrfull{jnf}} and turned out to be a \say{national project} and it was called \say{The village files}\citep{Pappe2006}. 


The \say{archive} was nearly complete in the late 1940s and the information became more specifically military-oriented. \citep{Pappe2006}. In 2018 the Israeli national archive was published online, but after a time of research through the archive in English and in Hebrew, those maps and the detailed information about the villages were not openly published. According to Shezaf (2019), The Secret Security Department (Malmab) of the Israeli Defense Ministry is responsible for concealing hundreds of documents as part of a concerted effort to conceal Nakba evidence \citep{Shezaf2019}.\say{Yehiel Horev, who headed Malmab for two decades, until 2007,
acknowledged to Haaretz that he launched the project, which is still
ongoing. He maintains that it makes sense to conceal the events of
1948, because uncovering them could generate unrest among the
country’s Arab population. Asked what the point is of removing
documents that have already been published, he explained that the
objective is to undermine the credibility of studies about the history
of the refugee problem}\citep{Shezaf2019}.


\subsection{Operation Ben-Ami}

Around 13 and 22 May 1948, Operation Ben-Ami was explicitly told in two phases that the villages had to be destroyed in retaliation for the convoy's defeat. Therefore the villages of Sumiriyya, Zib, Bassa, Kabri, Umm al-Faraj, and Nahr are exposed to an expanded, more brutal form of the Israeli force's \say{destroy-and-expel} exercise. They aimed to invade, execute, destroy and set fire to Kabri, Umm al-Faraj, and Nahr for the sake of occupation \citep{Pappe2006}. The operation aimed for capturing and clearing all the villages along the coast from the south of Acre to Ras Al-naqoura in the north by the Lebanese borders \citep{Morris2004, Morris2008}.

Part of the Haganna troops arrived with armored vehicles and some troops arrived by boat near Sumiriyya, the Haganna men went through the coast and they attacked Sumiriyya with mortars and left the eastern side open so that inhabitants could flee. \citep{Morris2004, Morris2008}.

Bassa took over a day to defeat because of the village militant's resistance and some volunteers from the \textit{\acrfull{ala}}. The resistance gave the Haganna another incitement to \say{punish} the village further than the expulsion of their people. Haganna troops stormed Bassa and demanded all the young men in front of one of the churches to be lined and then executed them. The last village that collapsed in phase two of the Ben-Ami campaign after the massacre was Al-Ghabisiyya \citep{Morris2004, Pappe2006}.  



\section{Al-Ghabisiyya Village}

Al-Ghabisiyya is 11.5 km north-east of the Acre. The village sat on a rocky heap rising from the Acre plain. The site was probably a large Canaanite town, extrapolating from the many caves that were used as tomb places.
Al-Ghabisiyya had 690 residents and, 125 houses in 1931. Built in the late 19th century, the village was populated by 150 people at the time \citep{Khalidi2015}.
\subsection{Al-Ghabisiyya before 1948}

 Al-Ghabisiyya has been surrounded by a wide range of trees such as olives, figs, and pomegranate. The village was close to Shaykh Dannun and Shaykh Dawud's other two villages. Shaykh Dawud and Shaykh Dannun were intersected at places, but they were only 500 meters away from Al Ghabisiyya. These villages total population was Muslim. 

The Ottomans had founded and built a school at Al-Ghabisiyya in 1886. The buildings of the village are constructed of reinforced concrete and in some situations, the rock kept together with mud or cement mortar. The village's economy is dependent on livestock farming, and the main harvests were grains, and vegetables. The villagers also cultivate olives and pressed it in two animal-drawn presses, one was in Al-Ghabissya and the other in Shaykh Dawud.

A total of 6633 dunums were distributed for grains from the lands of the three villages in 1944/45, 1371 dunums were replanted or used for fruit trees. That year, 300 dunums were committed to olive trees in Al-Ghabisyya \citep{Khalidi2015}.

\subsection{Occupation and Depopulation}

At the end of Operation Ben-Ami, the Haganah's invasion of Palestine's northwest corner, Al-Ghabisiyya fell. The operation, which started from 13 to 14 May 1948, was the last major Haganah offensive in Palestine before the end of the British Mandate \citep{Khalidi2015}.

In the words of the Israeli historian Benny Morris, \say{This was in line with Plan D provision for securing blocks of Jewish settlement even outside the partition plan borders} \cite[p.252]{Morris2004}.

The order was given to the Carmeli Brigade that carried out the operation on 19 May 1948, "to attack with the aim of conquest, the killing of adult males, destruction and torching villages of Al-Kabri, Umm al Faraj and Al-Nahr" \cite[p.253]{Morris2004}. The following night, 20-21 May, Al-Kabri was occupied as part of the second phase of Ben-Ami operation. Al-Nahr was captured during this second stage of the operation, along with several villages in western Galilee, north of Acre, between 20-21 May 1948. Carmeli Brigade units targeted Al-Ghabisiyya on the same date being the last village to be taken. Al-Ghabisiyya officially fell, the residents received the troops with white flags, but Carmeli's troops shot some residents and then executed six more \citep{Morris2008}. Most of its inhabitants were deported in the days or even weeks that followed \citep{Morris2004}.

From two locations, north and southeast, the attacks were carried out. The invading forces seized a home in the village's southernmost area, shelling the village out of the house, killing and wounding many of the civilians as they fled. The village militia preferred not even to fight the Zionist forces as they were too few (about twenty) and equipped very poorly. Most of those who were driven out remained in other villages in Galilee until the whole region fell at the end of October 1948, after that they were displaced to Lebanon \citep{Khalidi2015}.

Several people stayed until February 1949 in Al-Ghabisiyya. This time, a second expulsion by the military government took place during the same month on the pretext of \say{safety and order}, it's not clear where the villagers were expelled \citep{Morris2004}.

After the evacuation of Al-Ghabisiyya and its two adjacent villages of Shaykh Dannun and Shaykh Dawud, some of the residents of the other two villages were allowed by the Israeli government to return home. Just several families from Al-Ghabisiyya, Al-Nahr, Al-Tall, Umm Al-Faraj, Amqa, and Kuwaykat joined those who had not found refuge in Lebanon. Shaykh Dannun and Shaykh Dawud's two tiny villages were incorporated into a single village called Shaykh Dannun, with a population of about 1000 in 1973. The Al-Ghabisiyya village has not been repopulated \citep{Khalidi2015}.  

\subsection{The Village Today}

The only surviving landmark is the mosque, a building of brick a dome mounted above it, with arched doors and windows and internal elaborate arches. It's abandoned, the concrete plaster on the dome peels off, and the rest of the roof is covered by wild shrubs. It is easy to see the ruins of buildings, terraces, and the village cemetery among a thick forest of pine trees planted on the village site and most of the property. Netiv ha-Shayyara's settlement, implemented in 1950 by Iraqi Jewish immigrants, uses the nearby non-forest land for farming \citep{Khalidi2015}.    
\section{Current Political Context}

The West Bank and East Jerusalem became part of the Kingdom of Jordan after 1948 and the Gaza Strip became an Egyptian area of trust \citep{Houdaille2010}. Israel captured the whole of Palestine in 1967 and also Egypt's Sinai, and Syria's Golan Heights by a six-day battle. The Palestinian Resistance movements united under the name of Palestine Liberation Organization (PLO). The national uprising (in Arabic: intifada) started inside Palestine against the occupation in 1987, where it led to peace negotiations between Yitzhak Rabin representing the Israeli government and Yasser Arafat as a representative of the PLO in 1993. The Oslo Accords is the name of the negotiation between the Israeli government and the PLO, the Oslo Accords are focused on UN Security Council Resolution 242 recommending Israel's departure from the 1967 borders and the Palestinian people's right to self-determination. It was the expectation that a Palestinian state would be created since the Palestinians took control of some cities and lands from the West Bank and Gaza. But due to the Israeli policy of occupying more lands in the West Bank and committing violations of human rights on Palestinians that led to a second uprising (Intifada) in the year 2000 \citep{Shalhoub-Kevorkian2006}. The current situation in Palestine is a very controlled movement for Palestinians, a military checkpoint on every city entrance, and all the Palestinian areas are surrounded by a so-called separation wall. The separation wall is 3 times as long and twice the height of the Berlin wall \citep{Shalhoub-Kevorkian2006}.



\section{Cultural Identity}
 
The expulsion of the 1948 Palestinian village people did not affect a transient population, but an ancient ancestral farming community that belonged to a civilization that elevated the human culture with its connection to religion, literature, art, architecture, and science. Therefore, it is not impossible to imagine the intensity and toughness of the suffering which impacted the families that were displaced in 1948 or why their mental state was transferred in their diaspora to their grandchildren \citep{Khalidi2015}.
 
 The children have grown up fascinated by their relatives and grandparents tales of their abandoned homes and villages and the perfect view of the Palestinian environment and scenery they were forced to leave it behind them and flee.

They see they belong to Palestine through a fight for equality and equal rights for under-occupation Palestinians. The Palestinian connection has been turned into a bond that is more formal but critical. The bond that could be considered as Long-distance post-nationalism. The ongoing violence perpetrated on Palestinians has caused long-distance post-nationalism in the first place. That is the reaction to this abuse and inequality that is enhancing their engagement with their ancestral home country. It's not just about Palestine as the location of origin, it's more about Palestine as a cause. While the point of interaction with Palestine starts with a personal interest and personal story in many instances, it changes the engagement with Palestine from one that can be read nationally to one that can be viewed as entering a fight for equality in more general terms.
The degree to which long-distance post-nationalism will coexist with national ways of belonging in national politics can be hypothesized. Although many of the study researchers in the second generation sympathized with the Palestinian struggle for independence, none of them understood their connection to Palestine by direct involvement in nationalist movements or saw themselves as being reflected by the Palestinian Authority. Nevertheless, it is important to recognize how these various forms of connection are not inherently exclusive and may intersect with one another \citep{Blachnicka-Ciacek2018}.

It is necessary to see how the new post-nationalist conceptualization of Palestine by the second generation could have the ability to universalize the Palestinian struggle and, accordingly, make it inclusive and reachable to those Palestinians who may have felt distanced from their parent home country \citep{Blachnicka-Ciacek2018}. 

\section{The YALLAH! Hackathon}

\acrshort{yallah!} Is a student’s exchange research program between the University of Siegen in
Germany and Birzeit University in Palestine funded by the DAAD. The program was
constructed for a collaboration between students to create social innovation projects.
In \acrshort{yallah!} 2018 there were three different projects Mobile Makerspace, \acrshort{vr} Experience and
\acrshort{yallah!} Computer Club. The \acrshort{yallah!} program is divided into three phases: The preparation
phase (Bootcamp), a research phase in Germany, and an implementation phase in Palestine.
The \acrshort{yallah!} Hackathon preparation phase
the participants of \acrshort{yallah!} learned about some research methods and how to apply them in
the projects through a boot camp week. During the boot camp, the teams took their first
steps in the projects by planning the research track, setting a research question, and building
a strategy to achieve their goals.
The Palestinian and the German students attended the boot camp separately, therefore the
boot camp took a place once in Germany and another time in Palestine. The purpose of the
boot camp was to introduce and train the participants on the research methods (such as how
to make interviews, how to take field notes, and how to observe while working in the field).
The students split up into 3 groups each group studied one method through literature and
then presented it to the other participants. By the end of the presentation, the students
discussed the method together and practiced it inside the group. The boot camp was a very
good practice for the methods and emphasized the research skills of the participants. At the
boot camp in Germany, the Virtual reality project students made their first test in filming with
a 360o camera, they used the Samsung Gear 360 camera to record part of the boot camp
sessions in a purpose to use it later in the project as a testing video. Due to the average video
quality of the Samsung Gear 360 camera, the \acrshort{vr} students added a task to their list that is to
find a 360$^{\circ}$ camera with better quality. The students also discussed the interesting spots in
Palestine related to the historical and religious places mentioned in biblical stories.
The \acrshort{yallah!} Hackathon research phase
The Palestinian students visited Germany for the research phase. The Palestinian and the
German exchange students met in Siegen, Germany for the first time. The students gathered
on the first day and had an introduction from the supervisors about the whole \acrshort{yallah!}
program in general and the research phase in Germany. After the introduction on the first
day, the teams of each project gathered and discussed their plans for the research and the
implementation phases. The \acrshort{vr} team members discussed the kind of methods to be used in
the research, the team decided on Interviews and Thinking Aloud as methods for collecting
data from random people and from Palestinians who live in Germany. Therefore, multiple tasks needed to be done by the team to start working with those methods. The \acrshort{vr}
team split the tasks among the members: creating the interview questions guideline,
documentation, building a prototype, and logistics, for instance, to create a contact list of
interesting people to contact and have an interview with them.
After one month after the end of the research phase, the \acrshort{vr} team had their second gathering
in Palestine to start the implementation phase. The implementation phase contained a lot of
filming in different places in Palestine. Depending on the research results and the discussion
with the team members, they created an organized plan for visiting different places in
Palestine and capturing the area using the camera. The plan included the main cities of
Palestine, famous touristic sightseeing locations and religious places since people showed
high interest to see those spots. Therefore, the team recorded in Jerusalem, Jaffa, Haifa, Acre, Ramallah, Nablus, Jericho, Hebron, and Bethlehem.

\subsection{VR Application development}

During the \acrshort{yallah!} Hackathon trip the development on the \acrshort{vr} application faced a lot of challenges. It was challenging at the development level due to limited experience in \acrshort{vr} technology for the team. In Palestine, the challenges were more complicated and on different levels. The team faced difficulties in movement between cities, due to Israeli military checkpoints between the Palestinian cities. The checkpoints also exist between the Israeli areas and the Palestinian territories that are surrounded by the separation wall as well. Therefore, it reduced the recording time due to the traffic caused by checkpoints, the team couldn't film all the areas to avoid facing problems with the Israeli forces if they interpret filming as a threat for their security. Filming in touristic areas, wasn’t easy since there was a lot of tourists and the camera was standing on a unipod we had to stay around it to avoid it from falling. Al-Aqsa mosque wasn’t easy to film because it is open for tourists only for specific times, and the tourists can’t go inside the Mosque. Since only Muslims are allowed inside the mosque, the camera was taken by a Muslim student from the team and filmed inside the mosque. But there was an unfortunate event, and the camera fell on one of the lenses and gave a blurry picture. The team had the Samsung Gear 360 camera as a backup, and then it was used to film the whole mosque again. On the application development, Unity game engine had some downsides, it was a heavy weighted environment to develop on it a \acrshort{vr} platform, some computers didn’t handle it. But the team managed to work with it on a workstation. They arranged a workstation that could handle the \acrshort{vr} environment on Unity and the workstation had a good quality graphics card that was able to process all the graphics data in Unity. The other challenge was combining, editing and rendering the 360$^{\circ}$ videos. The high-quality videos that were recorded by the GoPro Fusion camera took a big capacity on the hard drives. First, the camera had two lenses, and each lens records a video on a different memory card. Therefore, the videos needed to be combined and after that processed. It was a team effort for solving this task it took a lot of time and computer power. The size of the16 videos was very big and the internet in Palestine was not good enough for uploading such videos over the Internet. Nevertheless, the application was not ready, it didn't have a \acrfull{ui} that would allow the user to navigate between the locations and cities. 

\chapter{Methodology}
\section{Interviews}
\section{Participatory Design and Design case studies in Virtual Reality}
\subsection{Definition of PD and Design case studies}
\subsection{Methods evaluation in ICT artifacts}



\chapter{Application Development}

\section{Hardware and Development Environments}
\subsection{Headsets}
\textbf{Google Cardboard}: The virtual reality platform was released in
2014 by Google. The platform is intended as a low-cost system to
encourage interest and development in VR applications. It was
named for its fold-out cardboard viewer. The Google Cardboard
headsets are built out of simple, low-cost components -
cardboard. Google open-sourced the schematics and the
assembly instructions freely on their site, allowing people to
assemble Cardboard themselves from readily available parts (“Google Cardboard,” 2014). The
cardboards were the best option to be used for the project due to the easy mobility and the
low price. It is easier to travel with it through airports or checkpoints since it’s cardboard.
\subsection{Cameras}
\textbf{GoPro Fusion}: the footage quality can reach up to 5.2K spherical video
resolution. The GoPro fusion can be controlled via a mobile
application through Bluetooth or Wi-Fi. The two lenses on the two
sides are not symmetrically aligned, they are off-axis. That helps to
process the images or the footage taken from the two lenses to not
have visible stitching or overlapping in the final image. That is a
common problem in most of the VR cameras to have a big overlapping on the final image. The
data is saved on two microSD cards one for each lens, the files need to be combined to have
a final 360o video (Easton, 2018). The camera was used in most of the project filming material,
it has the best footage quality and a perfect stabilization in the videos.


\textbf{Samsung Gear 360}: The small and rounded shape of the camera is ideal for
handheld shooting, although it has a socket also for a tripod. A small LCD
screen helps in navigating through the camera modes. The video resolution is
4K, while the still images are somewhat soft. The smartphone app is easy to
use and clear for the user also it offers a good range of viewing options. In
general is it a small and simple camera to use (Digital Camera, 2018). The
camera used as a backup camera during the project. The quality is acceptable
for a small and very light camera.

\subsection{Development Environments}

The VR technology is moving forward and there is an increasing number of tools and platforms
available for developers (“11 Tools for VR Developers,” 2017). VR technology has found its
way into different environments like computers, smartphone, and web. This section will
mention the two tools that were used by the VR team developers to build a mobile VR
application and a VR experience over the web. Nevertheless, most browsers are still struggling
with the headset device support. Most phones can be detected with the WebVR-polyfill and
if turned sideways, it will switch the dual display mode automatically that you can use Google
Cardboard or other headsets built for smartphones (“11 Tools for VR Developers,” 2017).


\textbf{Unity 3D}: Unity is one of the most famous game engines, it has a direct VR mode to preview
the work on any Head mounted display, which can be easier and faster for designers to boost
their productivity. Most of the Head mounted displays are supported in Unity. Unity works
with C\# and JavaScript; it is easy to learn due to the huge online community. Unity can export
the work to almost any platform even WebGL (“11 Tools for VR Developers,” 2017). Unity was
the best and most powerful tool to be used during the project to develop the VR experience.
Due to the easy implementation of the VR Environment in Unity, also the variety of platforms
that it allows the developer to distribute the software on it. In addition, there is a huge
community for Unity developers on the internet, where everyone can share knowledge and
expertise.


\textbf{A-frame}: A Mozilla open-source project allow you to develop and experience WebVR without
the need of learning Three.js or WebGL directly. This web framework built on top of Three.js
and WebGL to build virtual reality experiences by using an Entity-Component ecosystem in
HTML (“11 Tools for VR Developers,” 2017). A-Frame was the first choice for developing the
experience over the web due to its functionality and easy implementation on webpages. A
variety of examples are offered on A-Frame website, so the user can take those examples and
build on them or reuse their code on a project since it is an open-source platform.


\section{Development \& Implementation}

\subsection{Software}

\subsection{User Interface}
\chapter{Results}

\section{Preliminary Empirical Research Findings}
\section{On The Ground Research Findings}
\section{Reflection on UX Design}
\section{Evaluation of Glimpses from Palestine Application}
\chapter{Discussion}


\section{Discussion Results}




\section{Challenges}

As with every project, during development, there are several obstacles. I faced challenges during the development of the application and during my fieldwork. During the development process, I confronted difficulties in programming due to some functionalities like opening and closing panels for the user. I didn't have any prior knowledge about how to build and open a new interface for the user within VR. I developed the Application on Unity on the 2018 version, but that version was not compatible with the Plug-in that is responsible for building the terrain. To create the Al-Ghabisiyya terrain, I had to download Unity 2019 update. At the end of creating the final results, I combined the two projects in one file. That created a series of errors, and I needed to fix them. They were fixed and I installed the first copy of the application on my smartphone. Thus, I was able to do the usability test with it. 
Collecting the data about the villages, in general, it was a challenge. There wasn't a lot of sources about the details of each town. Maps were almost impossible to reach, even the ariel photos that were taken by the British Mandate.  The map that I received had no orientation or scaling key. Therefore, collecting the data and the maps to implement them and rebuild a village or a town it was very challenging. However, my connections from the field made it easier for me. I believe it would be even more difficult for someone who wasn't raised in Palestine or at least not from the same culture. I am Palestinian and I speak the language, and I found that challenging. To be able to collect data and insights the interviewee needs to trust the interviewer. Nevertheless, people in Palestine are not easy to trust anybody, especially when the conversation is about politics. Multiple people refused to be interviewed after I introduced them to the topic. They are from the West Bank refugee camps. There is no easy way to build trust with the refugees, they need to know you personally. Fortunately, I knew and interviewed refugees from the camps. I needed to interview also Palestinian refugees in the diaspora camps. The majority of Al-Ghabisiyya inhabitants were deported to Lebanon. I have no access to be in Lebanon physically, due to law restrictions on the Lebanese borders toward Palestinians. I have a contact in Lebanon, and in the refugee camps, she knew some people. She made contact with a man who once lived in Al-Ghabisiyya. That was another challenge, I was worried about the trust factor, but apparently, the man was happy about the interview and by talking to me while I am in Palestine. It was another behavior that could be influenced by the country's love and passion.  Al-Ghabisiyya is 4 hours away from where I lived in Jerusalem. I went there by train, it was the last train stop in the north of the country. It wasn't an easy trip therefore, my time there was limited. While I was there I had to conduct interviews and film the location in 360$^{\circ}$ degrees videos, before the time of the last train goes back to Jerusalem. One of the lenses in the camera did not function. I had to disassemble it and fix it while being there. Fortunately, it worked and I filmed the whole area successfully. 
\chapter{Limitation}
This chapter will address the limitations that I encountered during my research. There were technological constraints during the development of the application. And there were also restrictions on the ground, by collecting data and documentation. In the first section, I will talk about the technological boundaries and in the second section, I will display the restriction on the maps in Palestine. 
 

\section{Technology}

Developing the application with the large capacity of videos showed the limitations of my computer. The videos were rendered and prepared every night, but the consequences were to fill the hard drive with data. However, I saved all the videos on external hard drives and I made a daily backup for my laptop data.  At some point, my laptop stopped working, it didn't restart and not even load a safe mode. I tried all the possibilities but nothing worked. I started to restore the data from the backup drive. After retrieving all the data back, and replace the lost data. Unfortunately, the Android \acrfull{sdk} in Unity didn't function anymore. Therefore I couldn't install the new copy of the application on my smartphone. The copy would include Al-Ghabisiyya in \acrshort{vr} after building it, also a lot of other scenes from the country. Technology here limited me and time constraints as well because it was the last week for me in the field and I needed to let users test the Application. Luckily I had the previous copy of the application on my smartphone, and I showed the users the village on the laptop how did it look like. 
   

\section{Access to Maps}

The maps in Palestine are very unique, even with the new technologies that we have. Google maps, for instance, have a lack of information about the Palestinian areas at the West-bank. That also implies to those demolished villages there is no information related to them. The maps or even pictures for these villages are very complicated and very rare to find. As I mentioned in the Background chapter, the Israeli government blocked all this data from being published. This limited me to obtain more information and maps to reconstruct the other villages. The map of Al-Ghabissiya was taken by the British mandate from an ariel photo. If I had access to ariel photos or maps I would have constructed more villages in Virtual Reality. 
\chapter{Conclusion and Future Work}
\input{chapters/Conclusion.tex}


\bibliographystyle{apalike}
\bibliography{references}
\appendix
\chapter{Declaration of Consent}



\textbf{Project: Glimpses of Palestine}


Declaration of consent for sound, image and video recordings



I,-------------------------------------- , have been informed orally by Samer Shawar that sound, image and video recordings of me will be made as part of the above project.

The recordings serve exclusively to view and analyze the evaluation again. They are recorded with a recording device and then potentially written down by the staff of the research project.

There is a chance I am recognizable in the footage. For this reason, all persons involve in the evaluation are subject to absolute confidentiality. The recordings will not be passed on to third parties.

I am aware that my statements in scientific papers are quoted in excerpts, and I was assured that they will be treated anonymously. Image extracts, on which I am shown, are anonymized before a possible publication. All information that could lead to the identification of my person will changed or removed from the text.

Since I can potentially be recognized in the recordings, I have the right to have these recordings deleted at any time without any disadvantages. To get recordings of me deleted, I contact the researcher.

The declaration of consent for sound, image and video recordings is voluntary. I can revoke this declaration of consent at any time. In the event of rejection or withdrawal, I will not incur no costs or other disadvantages. However, participation in the study will then not be possible.

Personal contact data are stored separately from interview data and are inaccessible to third parties. After completion of the research project, I have been assured that my contact data will be automatically deleted, unless I expressly agree to further storage for the contact option for topic-related research projects. I can object to longer storage at any time.

Participation in the research in voluntary. I may at any time cancel an interview, refuse further participation and withdraw my consent to a recording and transcript without any disadvantage to me.

\vspace{5mm}
I agree to take part in an interview/several interviews, workshops, and focus groups as part of the research project mentioned above.


 $\Box$ Yes \hspace{15mm}
 $\Box$ No

\vspace{10mm}
I agree to be contacted for future related research projects. For this my contact details about the end of the research project remain stored.


 $\Box$ Yes \hspace{15mm}
 $\Box$ No


\vspace{10mm}
I have received a copy of this declaration of consent.


 $\Box$ Yes \hspace{15mm}
 $\Box$ No



\vspace{5mm}
For question or other concerns, I can contact the following: 


Samer Shawar


Kohlbettstrasse 15


D-570072 Siegen


samer.shawar@student.uni-siegen.de

\vspace{15mm}

--------------------------------------------------  \hspace{20mm}            ---------------------------------------------


Place, Date and Signature of Participant	\hspace{15mm}         	Name of Participant in Block letters
\vspace{10mm}

--------------------------------------------------  \hspace{20mm}            ---------------------------------------------


Place, Date and Signature of Researcher	\hspace{15mm}         	Name of Researcher in Block letters


\end{document}

