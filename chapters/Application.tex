
\section{Hardware and Development Environments}
\subsection{Headsets}
\textbf{Google Cardboard}: The virtual reality platform was released in
2014 by Google. The platform is intended as a low-cost system to
encourage interest and development in VR applications. It was
named for its fold-out cardboard viewer. The Google Cardboard
headsets are built out of simple, low-cost components -
cardboard. Google open-sourced the schematics and the
assembly instructions freely on their site, allowing people to
assemble Cardboard themselves from readily available parts (“Google Cardboard,” 2014). The
cardboards were the best option to be used for the project due to the easy mobility and the
low price. It is easier to travel with it through airports or checkpoints since it’s cardboard.
\subsection{Cameras}
\textbf{GoPro Fusion}: the footage quality can reach up to 5.2K spherical video
resolution. The GoPro fusion can be controlled via a mobile
application through Bluetooth or Wi-Fi. The two lenses on the two
sides are not symmetrically aligned, they are off-axis. That helps to
process the images or the footage taken from the two lenses to not
have visible stitching or overlapping in the final image. That is a
common problem in most of the VR cameras to have a big overlapping on the final image. The
data is saved on two microSD cards one for each lens, the files need to be combined to have
a final 360o video (Easton, 2018). The camera was used in most of the project filming material,
it has the best footage quality and a perfect stabilization in the videos.


\textbf{Samsung Gear 360}: The small and rounded shape of the camera is ideal for
handheld shooting, although it has a socket also for a tripod. A small LCD
screen helps in navigating through the camera modes. The video resolution is
4K, while the still images are somewhat soft. The smartphone app is easy to
use and clear for the user also it offers a good range of viewing options. In
general is it a small and simple camera to use (Digital Camera, 2018). The
camera used as a backup camera during the project. The quality is acceptable
for a small and very light camera.

\subsection{Development Environments}

The VR technology is moving forward and there is an increasing number of tools and platforms
available for developers (“11 Tools for VR Developers,” 2017). VR technology has found its
way into different environments like computers, smartphone, and web. This section will
mention the two tools that were used by the VR team developers to build a mobile VR
application and a VR experience over the web. Nevertheless, most browsers are still struggling
with the headset device support. Most phones can be detected with the WebVR-polyfill and
if turned sideways, it will switch the dual display mode automatically that you can use Google
Cardboard or other headsets built for smartphones (“11 Tools for VR Developers,” 2017).


\textbf{Unity 3D}: Unity is one of the most famous game engines, it has a direct VR mode to preview
the work on any Head mounted display, which can be easier and faster for designers to boost
their productivity. Most of the Head mounted displays are supported in Unity. Unity works
with C\# and JavaScript; it is easy to learn due to the huge online community. Unity can export
the work to almost any platform even WebGL (“11 Tools for VR Developers,” 2017). Unity was
the best and most powerful tool to be used during the project to develop the VR experience.
Due to the easy implementation of the VR Environment in Unity, also the variety of platforms
that it allows the developer to distribute the software on it. In addition, there is a huge
community for Unity developers on the internet, where everyone can share knowledge and
expertise.


\textbf{A-frame}: A Mozilla open-source project allow you to develop and experience WebVR without
the need of learning Three.js or WebGL directly. This web framework built on top of Three.js
and WebGL to build virtual reality experiences by using an Entity-Component ecosystem in
HTML (“11 Tools for VR Developers,” 2017). A-Frame was the first choice for developing the
experience over the web due to its functionality and easy implementation on webpages. A
variety of examples are offered on A-Frame website, so the user can take those examples and
build on them or reuse their code on a project since it is an open-source platform.


\section{Development \& Implementation}

\subsection{Software}

\subsection{User Interface}

“ In order to allow human-computer interaction it is necessary to use special interfaces designed to input a user's commands into the computer and to provide feedback from the simulation to the user” \citep{burdea2017virtual}.