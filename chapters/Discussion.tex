

\section{Discussion Results}




\section{Challenges}

As with every project, during development, there are several obstacles. I faced challenges during the development of the application and during my fieldwork. During the development process, I confronted difficulties in programming due to some functionalities like opening and closing panels for the user. I didn't have any prior knowledge about how to build and open a new interface for the user within VR. I developed the Application on Unity on the 2018 version, but that version was not compatible with the Plug-in that is responsible for building the terrain. To create the Al-Ghabisiyya terrain, I had to download Unity 2019 update. At the end of creating the final results, I combined the two projects in one file. That created a series of errors, and I needed to fix them. They were fixed and I installed the first copy of the application on my smartphone. Thus, I was able to do the usability test with it. 
Collecting the data about the villages, in general, it was a challenge. There wasn't a lot of sources about the details of each town. Maps were almost impossible to reach, even the ariel photos that were taken by the British Mandate.  The map that I received had no orientation or scaling key. Therefore, collecting the data and the maps to implement them and rebuild a village or a town it was very challenging. However, my connections from the field made it easier for me. I believe it would be even more difficult for someone who wasn't raised in Palestine or at least not from the same culture. I am Palestinian and I speak the language, and I found that challenging. To be able to collect data and insights the interviewee needs to trust the interviewer. Nevertheless, people in Palestine are not easy to trust anybody, especially when the conversation is about politics. Multiple people refused to be interviewed after I introduced them to the topic. They are from the West Bank refugee camps. There is no easy way to build trust with the refugees, they need to know you personally. Fortunately, I knew and interviewed refugees from the camps. I needed to interview also Palestinian refugees in the diaspora camps. The majority of Al-Ghabisiyya inhabitants were deported to Lebanon. I have no access to be in Lebanon physically, due to law restrictions on the Lebanese borders toward Palestinians. I have a contact in Lebanon, and in the refugee camps, she knew some people. She made contact with a man who once lived in Al-Ghabisiyya. That was another challenge, I was worried about the trust factor, but apparently, the man was happy about the interview and by talking to me while I am in Palestine. It was another behavior that could be influenced by the country's love and passion.  Al-Ghabisiyya is 4 hours away from where I lived in Jerusalem. I went there by train, it was the last train stop in the north of the country. It wasn't an easy trip therefore, my time there was limited. While I was there I had to conduct interviews and film the location in 360$^{\circ}$ degrees videos, before the time of the last train goes back to Jerusalem. One of the lenses in the camera did not function. I had to disassemble it and fix it while being there. Fortunately, it worked and I filmed the whole area successfully. 