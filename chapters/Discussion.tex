I will discuss the results and the challenges of the research in this chapter. The results have been collected during interviews in the pre-study phase in Germany and the Design phase in Palestine. Those results came along together after designing the application. Nevertheless, there were challenges during the research phases, I will mention them in detail in the second section of this chapter. 

\section{Discussion Results}
In this research, I wanted to check if Virtual Reality can let people overcome borders. During the pre-study, the results showed the need of the users and the connection that they have for the country.  Also, during \acrshort{yallah!}, the results were showing that people thought about Palestine that its only a war zone. The Palestinians in the diaspora also showed a lot of interest to know and see their country. They are highly connected to the country, but they also do not know how it looks like. The participants showed that they got their identity and their connection to Palestine only through their grandparent's and parent's stories. But also they got connected through the struggle cause of the Palestinians, and the unfairness that they face daily. Multiple links connect the people with the country. The grandparents who were deported out of Palestine, they have like a romantic relationship with Palestine. In the way that they described the country for their kids and grandchildren. It is described as heaven but the children would have different imaginations. That's the effect of why one of the participants told me that he didn't want to see the house that it belongs to his grandparents in the village because maybe it doesn't look like the image that he has for it in his head. Glimpses from Palestine give the ability for the user to experience the real-life in Palestine, from the struggle to the normal life in different places. I need to let people see what the real-life in Palestine, instead of being scared to go to Palestine and be exposed to danger. Glimpses from Palestine would be a safe platform to experience life safely. Also, it has the feature of time traveling for getting more historical facts and stories about how Palestine looked like before 1948. Al-Ghabisiyya is one of 536 villages, and in every village, there is a different story.  



\section{Challenges}

As with every project, during development, there are several obstacles. I faced challenges during the development of the application and during my fieldwork. During the development process, I confronted difficulties in programming due to some functionalities like opening and closing panels for the user. I didn't have any prior knowledge about how to build and open a new interface for the user within VR. I developed the Application on Unity on the 2018 version, but that version was not compatible with the Plug-in that is responsible for building the terrain. To create the Al-Ghabisiyya terrain, I had to download Unity 2019 update. At the end of creating the final results, I combined the two projects in one file. That created a series of errors, and I needed to fix them. They were fixed and I installed the first copy of the application on my smartphone. Thus, I was able to do the usability test with it. 
Collecting the data about the villages, in general, it was a challenge. There wasn't a lot of sources about the details of each town. Maps were almost impossible to reach, even the Ariel photos that were taken by the British Mandate.  The map that I received had no orientation or scaling key. Therefore, collecting the data and the maps to implement them and rebuild a village or a town it was very challenging. However, my connections from the field made it easier for me. I believe it would be even more difficult for someone who wasn't raised in Palestine or at least not from the same culture. I am Palestinian and I speak the language, and I found that challenging. To be able to collect data and insights the interviewee needs to trust the interviewer. Nevertheless, people in Palestine are not easy to trust anybody, especially when the conversation is about politics. Multiple people refused to be interviewed after I introduced them to the topic. They are from the West Bank refugee camps. There is no easy way to build trust with the refugees, they need to know you personally. Fortunately, I knew and interviewed refugees from the camps. I needed to interview also Palestinian refugees in the diaspora camps. The majority of Al-Ghabisiyya inhabitants were deported to Lebanon. I have no access to be in Lebanon physically, due to law restrictions on the Lebanese borders toward Palestinians. I have a contact in Lebanon, and in the refugee camps, she knew some people. She made contact with a man who once lived in Al-Ghabisiyya. That was another challenge, I was worried about the trust factor, but apparently, the man was happy about the interview and by talking to me while I am in Palestine. It was another behavior that could be influenced by the country's love and passion.  When I went to Al-Ghabisiyya It was a long trip therefore, my time there was limited. While I was there I had to conduct interviews and film the location in 360$^{\circ}$ degrees videos, before the time of the last train goes back to Jerusalem. One of the lenses in the camera did not function. I had to disassemble it and fix it while being there. Fortunately, it worked and I filmed the whole area successfully. 