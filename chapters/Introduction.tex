During the last decade the word ‘virtual’ became one of the most exposed words in the
English language, today we have virtual universities, virtual offices, virtual pets, virtual
studios, etc., and all of this because of virtual reality \acrshort{vr} (Vince, 2011). The \acrshort{vr} hit the
headline in the mid-1980’s and spawned a series of conferences, exhibitions, television
programs, and philosophical debates about the meaning of reality (Vince, 2011). “Virtual
reality is using a computer to create images of 3D scenes where the person can navigate and
interact” (Vince, 2011). Pinotti (2017) explained that immersivity and interactivity in virtual
environments are able to elicit in the user an intense feeling of “being there”, namely of being
embodied in an independent and self-referential world (Pinotti, 2017).

This master thesis will examine a virtual reality smart-phone application if it can let people overcome borders and if virtual reality can strengthen resilience for people by seeing their rebuilt villages in the virtual world. 


Virtual reality will need the development of the environment, according to the previous study at the Hackathon exchange program the users showed the interest of people for having a smart-phone application that allows them to experience the virtual reality at Palestine. For developing the application framework, a development environment like Unity Gaming engine will be used for improving the application due to the easy implementation of the Virtual Reality environment at Unity and the ability to distribute it over a variety of devices. Unity provides the ability to develop a virtual reality environment and Implement 3D models inside the real 360$^{\circ}$ videos.


Respectively for the maps and the data collected about the structure of the villages, the houses will be 3D modeled to rebuild a village or a part of it in the Virtual world. Blender will be used to create the 3D models of the houses and texture the buildings in the villages. Blender is an open-source 3D animation software, and it is structured for modeling and animation. The 3D modeled village will be integrated within the 360$^{\circ}$ video of the same village area. Points of interest will be distributed among the map of the village and the 360$^{\circ}$ video will be filmed from the same points of interest. The Villages actual locations will be filmed by a GoPro Fusion a 360$^{\circ}$ 5K camera. Through the prototype, the user will be able to see the actual location of the village nowadays, and will also have the option to see how was it pre-1948. 

The sound of the surrounding environment will immerse the user into the scenes, for interactivity between the user and the virtual environment, the user will have a gaze feature that allows commuting between the points of interest inside the village for instance. The gaze feature will be activated only when the user looks at the point of interest for a few seconds, a time-line will start moving to inform and prepare the user to be transferred to another scene. The user needs a virtual reality glasses to be able to get immersed in the application and see the whole environment. The virtual reality glasses are expensive and they need a very good computer to be able to operate them. Google created a VR lens out of cardboard, it allows the users to use their smart-phones as a display for the cardboard and experience virtual reality. The cardboard is easy to move with it and pass different checkpoints, airports, borders because it is only cardboard and it is also very cheap. That will make it easy for us to move around and show the application to different people and evaluate the application.

Participatory action research is best represented as a self-reflective
spiral of cycles of planning, acting and observing, reflecting and then re-planning in successive cycles of improvement \citep{KemmisSdanMcTaggart1988}. Participatory action research will allow us to evaluate the application with the participants and understand their concerns about their usability and interaction with the design. we will give the participants the prototype and let them test it, while we are observing how they interact and use the prototype, also we will let them reflect on the design and the product and take it for improvement in the design. Participatory action research will allow us to have a better evaluation from users to form a better action and interaction where the users had already conducted practice on the product. The evaluation will be taken by different participants from Palestinian refugees in Palestine, Jordan, and Lebanon, but also non-Palestinians will evaluate the application since the application is addressed for all people and see if Virtual reality can let them overcome borders.