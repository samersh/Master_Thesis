During the last decade the word ‘virtual’ became one of the most exposed words in the
English language, today we have virtual universities, virtual offices, virtual pets, virtual
studios, etc., and all of this because of virtual reality \acrshort{vr} \citep{Vince2011}. The \acrshort{vr} hit the
headline in the mid-1980’s and spawned a series of conferences, exhibitions, television
programs, and philosophical debates about the meaning of reality (Vince, 2011). “Virtual
reality is using a computer to create images of 3D scenes where the person can navigate and
interact” (Vince, 2011). Pinotti (2017) explained that immersivity and interactivity in virtual
environments are able to elicit in the user an intense feeling of “being there”, namely of being
embodied in an independent and self-referential world (Pinotti, 2017).

This master thesis will examine a virtual reality smart-phone application if it can let people overcome borders and if virtual reality can strengthen resilience for people by seeing their rebuilt villages in the virtual world. 


Virtual reality will need the development of the environment, according to the previous study at the Hackathon exchange program the users showed the interest of people for having a smart-phone application that allows them to experience the virtual reality at Palestine. For developing the application framework, a development environment like Unity Gaming engine will be used for improving the application due to the easy implementation of the Virtual Reality environment at Unity and the ability to distribute it over a variety of devices. Unity provides the ability to develop a virtual reality environment and Implement 3D models inside the real 360$^{\circ}$ videos.


Respectively for the maps and the data collected about the structure of the villages, the houses will be 3D modeled to rebuild a village or a part of it in the Virtual world. Blender will be used to create the 3D models of the houses and texture the buildings in the villages. Blender is an open-source 3D animation software, and it is structured for modeling and animation. The 3D modeled village will be integrated within the 360$^{\circ}$ video of the same village area. Points of interest will be distributed among the map of the village and the 360$^{\circ}$ video will be filmed from the same points of interest. The Villages actual locations will be filmed by a GoPro Fusion a 360$^{\circ}$ 5K camera. Through the prototype, the user will be able to see the actual location of the village nowadays, and will also have the option to see how was it pre-1948. 

The sound of the surrounding environment will immerse the user into the scenes, for interactivity between the user and the virtual environment, the user will have a gaze feature that allows commuting between the points of interest inside the village for instance. The gaze feature will be activated only when the user looks at the point of interest for a few seconds, a time-line will start moving to inform and prepare the user to be transferred to another scene. The user needs a virtual reality glasses to be able to get immersed in the application and see the whole environment. The virtual reality glasses are expensive and they need a very good computer to be able to operate them. Google created a VR lens out of cardboard, it allows the users to use their smart-phones as a display for the cardboard and experience virtual reality. The cardboard is easy to move with it and pass different checkpoints, airports, borders because it is only cardboard and it is also very cheap. That will make it easy for us to move around and show the application to different people and evaluate the application.

Participatory action research is best represented as a self-reflective
spiral of cycles of planning, acting and observing, reflecting and then re-planning in successive cycles of improvement \citep{KemmisSdanMcTaggart1988}. Participatory action research will allow us to evaluate the application with the participants and understand their concerns about their usability and interaction with the design. we will give the participants the prototype and let them test it, while we are observing how they interact and use the prototype, also we will let them reflect on the design and the product and take it for improvement in the design. Participatory action research will allow us to have a better evaluation from users to form a better action and interaction where the users had already conducted practice on the product. The evaluation will be taken by different participants from Palestinian refugees in Palestine, Jordan, and Lebanon, but also non-Palestinians will evaluate the application since the application is addressed for all people and see if Virtual reality can let them overcome borders.





The idea is to develop a new application to show Palestine with 360$^{\circ}$ videos, but it will include a part of one of the demolished villages before 1948. To rebuild one of those villages we need to know the structure of the village and how it looked like. The data about the structure of the village will be collected through personal stories, old pictures, British mandate aerial photos, and some maps made by a few people who lived in those villages. 
  
  
In the master project, a pre-study of the project will be conducted with Palestinians who live in the diaspora through interviews and workshops.  The Interview is a highly used method of collecting data in qualitative social research methods” \citep{Anyan2013}.  Kvale(1983)  described the purpose of as a method collection in social research as “... to gather descriptions of the life-world of the interviewee with respect to interpretation of the meaning of the described phenomena”  \citep{Kvale1983}. Semi-structured interviews will be used for collecting stories and exploring personal experiences about the Palestinians displacement and their relation with Palestine after living abroad. A questions guideline will be prepared for the interviews with  open-ended questions allowing a discussion with the interviewee rather than  formulated questions and answers.

After the interview some insights would be collected from the interviewee through a participatory design workshop. Participatory design (PD) is a set of theories, practices, and studies related to end-users as full participants in activities leading to software and hardware computer products and computer-based activities  \citep{Muller2003}. The field is extraordinarily diverse,  drawing on fields such as user-centered design, graphic design, software engineering, architecture, public policy, psychology, anthropology, sociology, labor studies, communication studies, and political science, and from localized experiences in diverse national and cultural contexts \citep{Gregory2003}.  Personal stories might give the ability for some refugees to draw the houses of their villages how they remember it. Applying a workshop for some of the Palestinian refugees to re-draw their houses would demonstrate a better image for the construction of the village, and that will give them the ability to tell us what they would like to see. houses, public places, walking paths between the houses, farms...etc. 
Information and communications technologies (ICT) artifacts should react to changing conditions of a social system \citep{Wulf2011}. Design case studies give us the ability to understand the relationship between social practice and the design of ICT artifacts that are built to support those practices, It is not so much about high tech, but about high-value \citep{Volker2013}. Design Case Studies ideally consist of three phases: (1) Analyze and particularly describe a social practice on a micro-level, as well describing the tools, media and their usage. (2) Describe the ICT artifacts from a product and from a process perspective, like the design concepts, and the applied design methods. (3) Document the introduction, appropriation and the potential re-design of the ICT artifact, documentation allows analyzing the trans-formative impact of certain functions and design options \citep{Volker2013}. Design case studies will allow us to check with the Palestinian refugees the tools or the practice that they do to see or to feel the connection with Palestine as a country and with their villages. After understanding their tools or practice, we can build a design depending on a specific design concept and methods. By letting them trying the prototype we will have general feedback about the interaction of the user with the App, and what would need to maintain and what kind of things needs to be re-designed. Design case studies will give us a general overview of the design concept and full fill us with the answers for the "how" and "why" about using the design methods.   
The Application will start by showing an overview of the whole map of historical Palestine. Points of interest will hover over the cities, the user will select a point of interest through the gaze feature. After selection, the map will be zoomed to the city location and will show more specified places within the city and the villages surrounding the city. The user will be transferred to a 360$^{\circ}$ video of the desired place, and inside the 360$^{\circ}$ video the user can commute inside the village for instance through the points of interest. sound of the surrounding environment will be added to the videos, as well as narrating the history and personal stories about the place. Within the demolished village, the user will have the ability to see how is the village look like now, and how it used to look like before 1948 by selecting an interactive button to travel in time. The user will have the ability to move back to the main map to choose to see different locations. All the methods that are used will help us to develop and collect the information and the data that is needed for the application and to be presented to the user.




The term virtual is becoming one of the most revealed words in the English language over the past decade, today we have virtual schools, virtual workplaces, virtual animals, virtual studios, etc \citep{Vince2011}.
A pre-study for a \acrfull{vr} smart-phone application that allows the users to visit Palestine without crossing any borders. Immersion and interactivity in virtual environments will give the user a profound feeling of \say{being there} that is, expressed in an individual and self-referential world (Pinotti, 2017). As a case study, Al-Ghabisiyya a Palestinian village that has been depopulated and demolished has been re-modeled and placed in the application. Therefore the user can travel back in time and see Palestine before 1948. Multiple platforms and applications are doing a similar purpose. Therefore I show a comparison between those platforms and applications to the application that I built. Then I will explain the research gap that I found through my pre-study, and how the application reduced the research gap. A detailed history timeline explained for understanding the setting of the research field and the problem that the application would help to solve. I will explain how this application developed from an idea that came out at a student exchange program \acrfull{yallah!} between the University of Siegen in Germany and the University of Birzeit in Palestine. The followed chapter will present the methodological approach of the research and how I conducted each method to collect data from the users and the field.  

