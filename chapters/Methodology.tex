In this section, I outline the methodological approach and the steps taken to form and adjust the concept of the design. 

\section{Design Case Studies}



\section{Interviews}

“The Interview is a highly used method of collecting data in qualitative social research methods” \citep{Anyan2013}. \cite{Kvale1983} described the purpose of the interview as a method of data collection in social research as “...to gather descriptions of the life-world of the interviewee with respect to interpretation of the meaning of the described phenomena” \cite[p.174]{Kvale1983}. Expert Interviews according to \cite{Christmann2016}  “in scientific research an individual is addressed as an expert because the researcher assumes – for whatever reason – that she or he has knowledge, which she or he may not necessarily possess alone, but which is not accessible to anybody in the field of action under study” \cite[p.18]{Christmann2016}. The team conducted 4 interviews. one of them was with a researcher who only heard about Palestine but never been there, and another one was a Palestinian who lives in Germany. The other two interviews were expert interviews, one expert in Virtual Reality and the other is an expert in the Palestinian historical and geography field. A questions guideline was made by the students to have a semi-structured interview with the participants, where they can collect a determined data, and on the other hand, the interview is open for additional input. 14 Ad-hoc interviews were conducted by the team with random people in the street, those were unstructured interviews and the students gave the participants the prototype to see the 360o video and give their opinion about what they saw in the video and their thoughts about Palestine.

\subsection{Expert Interview}



\section{Thinking aloud}

In digital environments, thinking aloud is widely used for usability testing \citep{VanWaes2000}. “The thinking-aloud method consists in having a user working with a computer system (prototype, paper mock-up or documentation) while 'thinking-aloud', i.e., spontaneously (or prompted) verbalizing ideas, facts, plans, beliefs, expectations, doubt, anxiety, etc. that comes to mind during the work” (Jørgensen, 1990). Thinking aloud is a method that presents a prototype to the participants in a scenario and often evaluated by the participants commenting on how they are using the prototype. The prototype was presented to the interviewees after they finish their interview with the students, it contained a 360o video of a bazaar in Bethlehem. The participants gave their comments and impressions about the prototype.


\subsection{Thinking Aloud in Virtual Reality}

 
\section{Field trip and Field notes}



