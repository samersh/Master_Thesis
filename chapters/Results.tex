The results of this research journey will be presented in this chapter. In the first section, I will introduce the preliminary findings during the pre-study in Germany. Following I will mention the findings during my visit to Palestine, obtained from interviews and the field. I will reflect on the user's reaction while using and exploring the application in \acrlong{vr}. The last part will show the user's evaluation for the application after using it, where would it be helpful and where not.


\section{Preliminary Empirical Research Findings}

In my pre-study in Germany, I conducted five interviews with multi-disciplinary people. However, most of them were a second and the third generation for refugee and immigrant Palestinians. I classified them as privileged Palestinian immigrants in Germany. Four of the interviewees went to Palestine on various occasions. The first one lived in a refugee camp in Gaza until he moved to Germany in the early 90's.  He used to visit Palestine annually until the year 2000. One day he was taken by his uncle to the village and showed him the village lands. As he said, there were stones from destroyed houses, old trees, and cactus. The village lies by the sea, on the coast of Majdal. Most of the people who lived there were farmers, while others were fishermen. The majority of the inhabitants became refugees in Gaza after the 1948 war. Some of the inhabitants immigrated for the second time to Jordan in the 1967 war. I asked him if he would like to see how his village looked like before 1948. He said, "I would love to, but I spent half of my life in Germany when I see my village where my ancestors used to live, I wouldn't have a strong connection honestly". In other words, he said that his relationship with Germany is better than with Palestine because something had changed in his life. His wife is German, and his children have nothing to do with Palestine even though he tells them about it. However, he would go back to his land to build a house for a summer vacation but not for living. His vision of the Palestinian refugee problem in the surrounding countries was to give the refugees full rights and citizenship as part of the solution. He said that the Palestinians, who live a good life in any place, would not abandon their lives and work and go back to Palestine. But the Palestinians, who live in poor living conditions in refugee camps, would go back to Palestine if they had a choice.
  


The second interviewee is a 53 years old man that was born in his village in Palestine before it got demolished. A small part of the village was captured by Israeli forces in 1948, and in 1967 it was fully occupied and exploded. He invited me to his home for the interview, the moment I entered the house, I felt I am back in Palestine. It was the ultimate Palestinian home from inside. He grew up in Jordan after they were deported out of the country. He came to Germany in the early '90s. He and his wife and children made their first visit to Palestine in 2017. He went to see the village, from the area he knew people to help him find it. The village was all lost and an Israeli settlement is now over it. There's no evidence for anything. He always heard about Palestine from his parents and grandparents. Therefore he always told his wife about it every day in the early days of his marriage. And then with the kids, he was telling them about his love for Palestine. He said, \say{The love of Palestine was raised in my kids by me and their mother}. He described the first moment when his son cried when he saw Jerusalem for the first time. He said, \say{It was like a dream for us, we couldn't believe we were there}. My children wanted to stay there, they were surprised by the hospitality over there. The people when they know that we came from Germany, they refuse to take money from us for their services, like rent or buying food. My children said, \say{If the Palestinians are like that, then there is a lot of things you did not tell us about it}. I asked him how his family told him about Palestine, he said, \say{I imagined all their stories about the village all the time, the houses the little store, the stones, their stories were very detailed}.


The third interviewee is 27 years old Student born for a Palestinian father and a German mother. The father is from Bait Lahyia in Gaza. He went to Gaza twice when he was a child. He remembers just a few pictures from there. The last time he visited Palestine, the first experience was profiling at the airport in Tel-Aviv. He was questioned for 5 hours. He defined himself as a German in his cultural identity but as a Palestinian in his political identity. The barrier between him and the culture was language, although he is aware of basic cultural things. But he is more connected to the political identity, because of his father talks about the Palestinian cause, then he read and researched about it a lot. He mentioned, "I felt an emotional connection when I visited Palestine, also with my memories and smells remembered from somewhere, it's like you can't explain it". Then he explained that he felt a very strong connection to the struggle. Due to his political connection with Palestine, he felt it as his people's struggle. He expressed his perspective about the connection to Palestine as follows " I think a connection to Palestine can persist over generations even in a diaspora and to be honest, if you see the refugee camps in Syria or Lebanon or anywhere basically, or in Germany where you don’t have camps, but the refugees are more spread over the country you can see they reconnect to their identity and still stick to it". He also pointed to the refugees in the camps, that their connection to Palestine is stronger than the privileged Palestinians lives in Germany. It is like a Utopian dream to go to Palestine for them. He explained that he wouldn't go and live in Palestine because he can't deal with living under military occupation. But if Palestine would gain its self-determination it would be an entirely different situation for him.



The next interview was with a guy for a Palestinian father and a Turkish mother. His father lived in a refugee camp in Jordan until he became 20, he immigrated to Germany. The father didn't push him into the Palestinian culture and not to the Arabic language. His father wanted him to be German. Even though he started to research for his second identity. Since his early age, his Turkish side was stronger, because he spoke the language and went to Turkey every year. Last year was a breakout for him because he traveled to Palestine. "As much as you read about Palestine, or see media, you cannot imagine what is going on or how it feels or how the people feel until you visit it, and after that, I felt a strong bond, strong connection to the country, to the people and the whole situation." The political situation made him see how people are suffering, and he wanted to change something, he said: "they are friendly very welcoming, and you just want to do something for them, they are your people more or less." He explained his visit that he didn't experience things that tourists would experience, instead, he went deep inside and experienced the daily life, the feelings and emotions are stronger, you don't forget them as he noted. He explained to me when he saw the political situation on media, it is not touching as real. He clarified that there should be people visiting and experiencing life,  for making a difference in the situation over there. For his identity, he said that he is a mix of the German, Turkish, and Palestinian culture.  He also explained that it is possible to come with one cultural identity from three influenced cultures, and he assumed that there is a lot of people like him. Because Germany is his home he does not see himself living in Palestine. He does not even speak Arabic, but he would like to visit Palestine as much as possible.

The last interview conducted with  24 years old student, she was born in Germany for Palestinian parents who lived in refugee camps in Lebanon. She has never been to Palestine. She identifies herself as a Palestinian. However, most of her family lives in Lebanon, and she has no relatives in Palestine. Her Palestinian identity grew up in her since she is a child. She noted that since the parents love their country, they would make their kids love it as well. Then she explained that she didn't get that only from her family, but she was always searching pm the internet to know more about Palestine. She said: "observing the unfairness that happens over there, it makes you feel that there is a connection with those people even if you are not a Palestinian, you feel like you should do something to make the situation get better for those people." She would love to travel to Palestine, but she doesn't prefer to pass through Israel. Her father always described for her the living in Palestine is different even the air is fresh, and the atmosphere is different. Although he didn't visit Palestine before, he narrates it as how it was described for him. Everything over there is natural and organic, and still has a beautiful soul in it. "For me, I see Palestine differently, do you know how is heaven described to you?, I see it like that. I even have this idea that people in Palestine are happier than us here in Germany, they have a simple life, and they value everything they get or achieve, here in Germany, everything is normal". With the heavenly vision about Palestine, she said that she wouldn't live there. Maybe she would try for a short period, but she is used to the life here in Germany and used to the mentality here not to the one in Palestine. However, she would like to do and help with anything for Palestine.

\section{On The Ground Research Findings}

I conducted five interviews during my field trip to Palestine, with a variety of people in different locations. I interviewed two people who were living in Al-Ghabisiyya before 1948. But one of those interviews conducted through Skype, since the man is a refugee in Lebanon. Then there were two more interviews in the West bank with two Palestinian refugees. The last interview I did with a Palestinian refugee from Lebanon at the airport, on my way back to Germany, and now he's living in Germany. Throughout the first interview, I was introduced to the events of the day that Al-Ghabisiyya got depopulated. One of Al-Ghabisiyya community representatives agreed with the Haganah to not attack the village, and the Haganah will not face resistance. They agreed to have a sign that someone would wave with a white flag up the mosque. Yet, when the Haganah troops arrived at Al-Ghabisiyya a man from the village went on the mosque roof and waved with the white flag, but the Hagana troops shot him dead, the people had panic and they fleed of the village. In that attack, 11 people were killed and most of the inhabitants were evacuated to Lebanon. The inhabitants who stayed in Palestine they moved to other villages where they have relatives. Months later the man who conducted the agreement with the Haganah intelligence met them again blaming them for what happened. They replied mildly, \say{We are sorry, but the captain of the unit, didn't know about our agreement, but the people can go back to Al-Ghabisiyya}. Half of the people who used to live there were already in Lebanon, but the ones who remained in Palestine, they went back to the village and lived there from the end of 1948 until January 1950, then the army came back again and gave them 48 hours to leave the village, and who stays there they would be deported to Lebanon. Later, Al-Ghabisiyya people made a lot of demonstrations, which led to raising the cause to the higher council court. The court decision was to let all the people live back in the village with freedom of movement. When the people tried to go back and get into the village, the army blocked the road and didn't allow them to pass due to order came from the Israeli government saying that Al-Ghabisiyya is a military closed zone. In 1956 they demolished all the houses and ran over them by bulldozers, then they planted trees so they hide the evidence of the demolished houses.  The mosque was kept just to say that they preserve holy places, as giving a good image for the apartheid. In another village the mosque was demolished at night, the people woke up on the second day and didn't find the mosque, not even the stones, it was demolished and the stones were taken as well. 


In the second interview, it was conducted through Skype with a Palestinian refugee from Al-Ghabisiyya. He told me that he had a permit to visit his relatives from the Israeli government in 1973, and when he visited Al-Ghabisiyya he saw all the village is demolished but he found his figs tree from his home is still standing, \say{It was the tastiest figs that I've ever had in my life}. Then he told me about how they used to live a very simple life, the school was in front of the mosque, they used to study mathematics, reading, and Holy Quran. They were getting drinking water from the near villages since they had water springs. He explained that the buildings of the house were similar to each other, they were all one-floor building. The families and relatives were living near each other, and other families live further. He used to attend the school in Acre after he finished his studies in Al-Ghabisiyya but after they were deported to Lebanon,  he started to work in a steel factory since he was 12 years old. He didn't give me more details about the village, I tried to get details about the buildings how they look like in details but it was very difficult due to the call over Skype. 


The other two interviews were conducted with Palestinian refugees in the WestBank. Both of them are originally from villages near Jaffa, one of them told me about his story with his 5 years old son, he took him once to Jaffa and he showed it to him and explained that he is originally from this city, he smoothly narrated the story, to not shock the kid by the facts, but just to introduce him to his identity, and his origins. After they were swimming in the sea and played on the beach. The kid told his father at once: \say{Your city, I mean our city is beautiful}. That was the moment where the kid has received his identity.  When he took him to his mother and he asked him to tell her the same sentence, the mother got emotional and cried. The man asked her not to cry in front of the kid, but she replied: \say{The kid should see and understand, the value of the homeland for us}. The other interview the man told me that their house is still standing in the village where they were deported from, it belongs to his grandparents, he never went there. His grandmother before they leave the house, she took the luxury culinary and she hid them under the apple tree. She thought it would be two days and they will come back home, so she hid the culinary so nobody steals them, she didn't think that they will steal the whole country. At the beginning of the 2000s, his uncle went to the house to visit it, it's occupied by Iraqi Jewish family now. They let him in, he dug under the apple tree and he pulled out his mother culinary that was still hidden. I asked the participant why he didn't go to see the house. He replied: \say{I don’t want to see our house, I like to keep the image in my head, I cannot handle it emotionally and psychologically}.  

I conducted the last interview at the airport with a Palestinian who was a refugee in Lebanon and now he is living in Germany. He told me that he always heard the stories from his parents about the village. The family was separated, some stayed in Palestine, some went to Lebanon and others went to Jordan. During the civil war in Lebanon, he decided to move to Germany. After a long time of searching and writing, the first time he got in touch with his family and relatives in Palestine was in 2014, he went to see the village but all of it is demolished, there was a mosque still standing but nowadays its surrounded by a metal siege and the mosque is not there anymore. He told me everyone would like to see how their villages look like and even go back to Palestine, but it is not possible.  I asked him if there was a chance for him to go and live in Palestine, he said: \say{I would leave Germany and everything to live in Palestine, there is nothing better than living in your homeland}.

During my visit to the field, the first time I went on a hiking trip in the surrounding area of Al-Ghabisiyya the area is a naturally beautiful landscape, and full of natural resources, I passed on several water springs, and I saw three Israeli water stations pumping all the water to the settlements only. Even though there is still some water left and I was walking most of the time in the water. The temperature was 36-38 Celsius and on the route that I took, there was a pool of water in one of the stones of the mountain. I wanted to cool down, but I couldn't get into the water since it was very cold. The nature is incredible mountains, forests, and water springs. One of the water springs it was a crystal clear water it was safe to drink from it, and I filled my bottle from that spring. On another day I went to Al-Ghabisiyya, I took the train from Jerusalem's main station, it was almost three and a half hours trip until I arrived at the station of Nahariya it is the last station in the north.  After that, I took the bus and I told the bus driver, \say{Drop me by Al Ghabisiyya}, he replied: \say{you mean Junction ben Alamein}, I smiled and said \say{exactly, Al Ghabisiyya}, I was following the bus route on Google maps to know, when I get closer to the village. Once I arrived, I looked around me, I saw that I am in a high way, no one can see or sense that there is a village there, it's full of trees. The only clear thing that there is a newly build Jewish cemetery. I crossed the high way and went to the junction that leads to the cemetery. The ones who pass from that junction is either industrial trucks or cars that are going to the cemetery. I walked up the mountain to the location of Al Ghabisiyya, and there I saw a sign says: \say{Welcome to Al Ghabisiyya} and near it, other signs tell more details about the village. I went more up, and I saw the mosque surrounded by a metal siege. Everything else around it is just trees and the stones of the other houses that were demolished. 

\section{Reflection on UX Design}

During my research and after each interview I conducted, I gave the users the chance to try and to check the application prototype, using the Thinking Aloud method. Meanwhile, they were testing it, I was observing and taking notes on their reaction and behavior. Even without interviews, I tested the prototype with people of different ages. The test was conducted with the participants while they are sitting on a chair and in a comfortable condition. 
I noticed that most of the participants were fully immersed. They were checking left, right, up and down, they also turned around. I realized some of them had a positive emotional feeling, I noticed the excitement and passion that they want to explore. The prototype contained videos from Jerusalem, one of the participants, is a resident of Jerusalem but he was excited about every scene. VR also changed the perspective of a participant, during the interview she said \say{I don't think I would live in Palestine}, while she is using the application, she said, \say{I think I would change my mind about not living in Palestine}. Hence, VR can increase emotional responses, and generate stronger emotions with higher values more than regular display settings \citep{Estupinan2014CanStudy}. The participants engaged in the virtual environment quickly, surprisingly they started to navigate into different locations, and notice a lot of details. \acrshort{vr} could raise the level of attention for participants \citep{Estupinan2014CanStudy}. It was easy for Participants to get engaged in the application, the users were in ages between 17-78 years old, and all of them used the application smoothly. The notes that I took during observing the participants, were showing that the design achieved the immersion, interaction, and imagination of the Virtual Reality. 
 

\section{Evaluation of Glimpses from Palestine Application}

I asked the participants to evaluate the application after each interview and what they thought about it. I have received various perspectives and outcomes. For the participants, it was easy for them to use the application. Regarding the interface, it wasn't complicated, but they preferred to have a slide of instructions at the beginning to understand what each button does, Even though, a participant said: \say{It is great, and the good thing about it that I don't use my hands for pushing buttons}. One participant said: \say{It's almost real, I just need to see and smell what I see there}. The technical evaluation was positive. But, when I asked about the value of the application. I received different kinds of results. The first evaluation was \say{The application is great, but that cannot replace our right of return}. This was a sign for me in my first interview that I should explain more for the participants that this project is not intended to replace the reality but to encourage people to explore Palestine and learn more without crossing borders. 
A participant described Glimpses from Palestine as an application that is searching for the value of the country into Palestinians, and that is what great about it. It is reminding the Palestinians, that our country is a value and we need to preserve this value since we do not have the power to preserve the country as a material.  He continued saying: \say{The most people who would highly interact with this application are the refugees in the diaspora because of its emphasizing the value of the country in them even more}. This thought was confirmed by a Palestinian refugee who used to live in Lebanon, he said: \say{The refugees in Lebanon would love to see Palestine on reality but with this application, it would be fantastic for them, they would highly interact with it}. One of the users said, \say{Everyone would like to try this application, and it is interesting, but what would happen after they see it? it is not the same as you go there}. And added, \say{Are the results worth the effort of rebuilding the villages?}. While another participant noted that he was nervous before going to Palestine. \say{An application like this would have prepared me for what I am going to see in Palestine}. He explained that Palestine is not an easy country, it has a conflict, and a lot of people are scared to go there. \say{The application can show normal life and also what the people are suffering from, both are important}. 
The evaluation hypothesis showed that Glimpses from Palestine application can make an impact on different levels, for Palestinians in the diaspora, and for non-Palestinians who are curious to know more and see a fair image for Palestine. Nevertheless, It does need a wider spectrum of work, for more goals and more significant results, that will be explained in the future work section.
