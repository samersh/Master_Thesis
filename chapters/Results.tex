The results of this research journey will be presented in this chapter. In the first section, I will introduce the preliminary findings during the pre-study in Germany. Following I will mention the findings during my visit to Palestine, obtained from interviews and the field. I will reflect on the user's reaction while using and exploring the application in \acrlong{vr}. The last part will show the user's evaluation for the application after using it, where would it be helpful and where not.


\section{Preliminary Empirical Research Findings}

In my pre-study in Germany, I conducted five interviews with multi-disciplinary people. However, most of them were a second and the third generation for refugee and immigrant Palestinians. I classified them as privileged Palestinian immigrants in Germany. Four of the interviewees went to Palestine on various occasions. The first one lived in a refugee camp in Gaza until he moved to Germany in the early 90's.  He used to visit Palestine annually until the year 2000. One day he was taken by his uncle to the village and showed him the village lands. As he said, there were stones from destroyed houses, old trees, and cactus. The village lies by the sea, on the coast of Majdal. Most of the people who lived there were farmers, while others were fishermen. The majority of the inhabitants became refugees in Gaza after the 1948 war. Some of the inhabitants immigrated for the second time to Jordan in the 1967 war. I asked him if he would like to see how his village looked like before 1948. He said, "I would love to, but I spent half of my life in Germany when I see my village where my ancestors used to live, I wouldn't have a strong connection honestly". In other words, he said that his relationship with Germany is better than with Palestine because something had changed in his life. His wife is German, and his children have nothing to do with Palestine even though he tells them about it. However, he would go back to his land to build a house for a summer vacation but not for living. His vision of the Palestinian refugee problem in the surrounding countries was to give the refugees full rights and citizenship as part of the solution. He said that the Palestinians, who live a good life in any place, would not abandon their lives and work and go back to Palestine. But the Palestinians, who live in poor living conditions in refugee camps, would go back to Palestine if they had a choice.
  


The second interviewee is a 53 years old man that was born in his village in Palestine before it got demolished. A small part of the village was captured by Israeli forces in 1948, and in 1967 it was fully occupied and exploded. He invited me to his home for the interview, the moment I entered the house, I felt I am back in Palestine. It was the ultimate Palestinian home from inside. He grew up in Jordan after they were deported out of the country. He came to Germany in the early '90s. He and his wife and children made their first visit to Palestine in 2017. He went to see the village, from the area he knew people to help him find it. The village was all lost and an Israeli settlement is now over it. There's no evidence for anything. He always heard about Palestine from his parents and grandparents. Therefore he always told his wife about it every day in the early days of his marriage. And then with the kids, he was telling them about his love for Palestine. He said, \say{The love of Palestine was raised in my kids by me and their mother}. He described the first moment when his son cried when he saw Jerusalem for the first time. He said, \say{It was like a dream for us, we couldn't believe we were there}. My children wanted to stay there, they were surprised by the hospitality over there. The people when they know that we came from Germany, they refuse to take money from us for their services, like rent or buying food. My children said, \say{If the Palestinians are like that, then there is a lot of things you did not tell us about it}. I asked him how his family told him about Palestine, he said, \say{I imagined all their stories about the village all the time, the houses the little store, the stones, their stories were very detailed}.


The third interviewee is 27 years old Student born for a Palestinian father and a German mother. The father is from Bait Lahyia in Gaza. He went to Gaza twice when he was a child. He remembers just a few pictures from there. The last time he visited Palestine, the first experience was profiling at the airport in Tel-Aviv. He was questioned for 5 hours. He defined himself as a German in his cultural identity but as a Palestinian in his political identity. The barrier between him and the culture was language, although he is aware of basic cultural things. But he is more connected to the political identity, because of his father talks about the Palestinian cause, then he read and researched about it a lot. He mentioned, "I felt an emotional connection when I visited Palestine, also with my memories and smells remembered from somewhere, it's like you can't explain it". Then he explained that he felt a very strong connection to the struggle. Due to his political connection with Palestine, he felt it as his people's struggle. He expressed his perspective about the connection to Palestine as follows " I think a connection to Palestine can persist over generations even in a diaspora and to be honest, if you see the refugee camps in Syria or Lebanon or anywhere basically, or in Germany where you don’t have camps, but the refugees are more spread over the country you can see they reconnect to their identity and still stick to it". He also pointed to the refugees in the camps, that their connection to Palestine is stronger than the privileged Palestinians lives in Germany. It is like a Utopian dream to go to Palestine for them. He explained that he wouldn't go and live in Palestine because he can't deal with living under military occupation. But if Palestine would gain its self-determination it would be an entirely different situation for him.



The next interview was with a guy for a Palestinian father and a Turkish mother. His father lived in a refugee camp in Jordan until he became 20, he immigrated to Germany. The father didn't push him into the Palestinian culture and not to the Arabic language. His father wanted him to be German. Even though he started to research for his second identity. Since his early age, his Turkish side was stronger, because he spoke the language and went to Turkey every year. Last year was a breakout for him because he traveled to Palestine. "As much as you read about Palestine, or see media, you cannot imagine what is going on or how it feels or how the people feel until you visit it, and after that, I felt a strong bond, strong connection to the country, to the people and the whole situation." The political situation made him see how people are suffering, and he wanted to change something, he said: "they are friendly very welcoming, and you just want to do something for them, they are your people more or less." He explained his visit that he didn't experience things that tourists would experience, instead, he went deep inside and experienced the daily life, the feelings and emotions are stronger, you don't forget them as he noted. He explained to me when he saw the political situation on media, it is not touching as real. He clarified that there should be people visiting and experiencing life,  for making a difference in the situation over there. For his identity, he said that he is a mix of the German, Turkish, and Palestinian culture.  He also explained that it is possible to come with one cultural identity from three influenced cultures, and he assumed that there is a lot of people like him. Because Germany is his home he does not see himself living in Palestine. He does not even speak Arabic, but he would like to visit Palestine as much as possible.

The last interview conducted with  24 years old student, she was born in Germany for Palestinian parents who lived in refugee camps in Lebanon. She has never been to Palestine. She identifies herself as a Palestinian. However, most of her family lives in Lebanon, and she has no relatives in Palestine. Her Palestinian identity grew up in her since she is a child. She noted that since the parents love their country, they would make their kids love it as well. Then she explained that she didn't get that only from her family, but she was always searching pm the internet to know more about Palestine. She said: "observing the unfairness that happens over there, it makes you feel that there is a connection with those people even if you are not a Palestinian, you feel like you should do something to make the situation get better for those people." She would love to travel to Palestine, but she doesn't prefer to pass through Israel. Her father always described for her the living in Palestine is different even the air is fresh, and the atmosphere is different. Although he didn't visit Palestine before, he narrates it as how it was described for him. Everything over there is natural and organic, and still has a beautiful soul in it. "For me, I see Palestine differently, do you know how is heaven described to you?, I see it like that. I even have this idea that people in Palestine are happier than us here in Germany, they have a simple life, and they value everything they get or achieve, here in Germany, everything is normal". With the heavenly vision about Palestine, she said that she wouldn't live there. Maybe she would try for a short period, but she is used to the life here in Germany and used to the mentality here not to the one in Palestine. However, she would like to do and help with anything for Palestine.

\section{On The Ground Research Findings}

I conducted five interviews during my field trip to Palestine, with a variety of people in different locations. I interviewed two people who were living in Al-Ghabisiyya before 1948. But one of those interviews conducted through Skype, since the man is a refugee in Lebanon. Then there were two more interviews in the West bank with two Palestinian refugees. The last interview I did with a Palestinian refugee from Lebanon at the airport, on my way back to Germany, and now he's living in Germany.



\section{Reflection on UX Design}

During my research and after each interview I conducted, I gave the users the chance to try and to check the application prototype, using the Thinking Aloud method. Meanwhile, they were testing it, I was observing and taking notes on their reaction and behavior. Even without interviews, I tested the prototype with people of different ages. The test was conducted with the participants while they are sitting on a chair and in a comfortable condition. 
I noticed that most of the participants were fully immersed. They were checking left, right, up and down, they also turned around. I realized some of them had a positive emotional feeling, I noticed the excitement and passion that they want to explore. The prototype contained videos from Jerusalem, one of the participants, is a resident of Jerusalem but he was excited about every scene. VR also changed the perspective of a participant, during the interview she said \say{I don't think I would live in Palestine}, while she is using the application, she said, \say{I think I would change my mind about not living in Palestine}. Hence, VR can increase emotional responses, and generate stronger emotions with higher values more than regular display settings \citep{Estupinan2014CanStudy}. The participants engaged in the virtual environment quickly, surprisingly they started to navigate into different locations, and notice a lot of details. VR could raise the level of attention for participants \citep{Estupinan2014CanStudy}. It was easy for Participants to get engaged in the application, the users were in ages between 17-78 years old, and all of them used the application smoothly. The notes that I took during observing the participants, were showing that the design achieved the immersion, interaction, and imagination of the Virtual Reality. 
 

\section{Evaluation of Glimpses from Palestine Application}

I asked the participants to evaluate the application after each interview and what they thought about it. I have received various perspectives and outcomes. For the participants, it was easy for them to use the application. Regarding the interface, it wasn't complicated, but they preferred to have a slide of instructions at the beginning to understand what each button does, Even though, a participant said: \say{it is great, and the good thing about it that I don't use my hands for pushing buttons}. One participant said: \say{It's almost real, I just need to see and smell what I see there}. The technical evaluation was positive. But, when I asked about the value of the application. I received different kinds of results. The first evaluation was \say{The application is great, but that cannot replace our right of return}. This was a sign for me in my first interview that I should explain more for the participants that this project is not intended to replace the reality but to encourage people to explore Palestine and learn more without crossing borders. 
A participant described Glimpses from Palestine as an application that is searching for the value of the country into Palestinians, and that is what great about it. It is reminding the Palestinians, that our country is a value and we need to preserve this value since we do not have the power to preserve the country as a material.  He continued saying: \say{The most people who would highly interact with this application are the refugees in the diaspora because of its emphasizing the value of the country in them even more}. This thought was confirmed by a Palestinian refugee who used to live in Lebanon, he said: \say{The refugees in Lebanon would love to see Palestine on reality but with this application, it would be fantastic for them, they would highly interact with it}. One of the users said, \say{Everyone would like to try this application, and it is interesting, but what would happen after they see it? it is not the same as you go there}. And added, \say{Are the results worth the effort of rebuilding the villages?}. While another participant noted that he was nervous before going to Palestine. \say{An application like this would have prepared me for what I am going to see in Palestine}. He explained that Palestine is not an easy country, it has a conflict, and a lot of people are scared to go there. \say{The application can show normal life and also what the people are suffering from, both are important}. 
The evaluation hypothesis showed that Glimpses from Palestine application can make an impact on different levels, for Palestinians in the diaspora, and for non-Palestinians who are curious to know more and see a fair image for Palestine. Nevertheless, It does need a wider spectrum of work, for more goals and more significant results, that will be explained in the future work section.
